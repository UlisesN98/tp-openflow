\hypersetup{
    pdfauthor={Camila General, Santiago	Vaccarelli, Federica Rodríguez Negri, Facundo Kresta y Ulises Nieva},
    pdftitle={TP N◦1: File Transfer},
    pdfsubject={Transferencia de archivos y protocolos de comunicación},
    pdfkeywords={transferencia de archivos, protocolos, redes, comunicación, UDP, TCP, Stop and Wait, Selective Repeat}  
}
\linespread{1}
\setlist[enumerate,1]{label=\arabic*., topsep=0pt}

\pagestyle{fancy}
\setlength{\headheight}{15pt}

\addto\extrasspanish{%
  \def\figureautorefname{Figura}%
  \def\tableautorefname{Cuadro}%
  \def\equationautorefname{Ecuación}%
  \def\sectionautorefname{Sección}%
  \def\subsectionautorefname{Subsección}%
}

\setlength{\tabcolsep}{16pt}

\bibliographystyle{unsrt}

\lhead{TP N°1: File Transfer}
\rhead{Redes - FIUBA}
\cfoot{\thepage}

\definecolor{LightGray}{gray}{0.9}

\lstset{
    basicstyle=\ttfamily\small,       % Estilo básico del texto
    frame=single,                     % Caja alrededor del texto
    captionpos=b,                     % Posición del caption (abajo)
    breaklines=true,                  % Permitir saltos de línea
    showstringspaces=false            % No mostrar espacios en strings
    showstringspaces=false,           % No mostrar espacios en strings
    numbers=left,                     % Números de línea a la izquierda
    numberstyle=\footnotesize,                % Estilo de los números de línea
    stepnumber=1
}


% Entorno para bloques de código
\lstnewenvironment{codelisting}[1][]{%
    \refstepcounter{codelisting} % Incrementa el contador de codelisting
    \lstset{
        numbers=left,                     % Números de línea a la izquierda
        numberstyle=\footnotesize,        % Estilo de los números de línea
        basicstyle=\ttfamily\small,       % Estilo básico del texto
        caption={Código \thecodelisting: ##1}, % Caption personalizado
        label={code:\thecodelisting},     % Etiqueta para referencias
        #1                                % Permitir opciones personalizadas
    }
}{}

% Definimos un nuevo entorno flotante "log"
\newfloat{log}{htbp}{lop}[section]
\floatname{log}{Log}

% Definimos un nuevo entorno flotante "secuencia"
\newfloat{secuencia}{htbp}{lop}[section]
\floatname{secuencia}{Secuencia}

% Definimos un nuevo entorno flotante "secuencia"
\newfloat{diagrama}{htbp}{lop}[section]
\floatname{diagrama}{Diagrama}

% % Definimos entorno Log con numeración por sección
% \newtcolorbox[auto counter, number within=section]{logbox}[2][]{%
%     enhanced,
%     breakable,
%     colback=white,
%     colframe=black,
%     fonttitle=\bfseries,
%     title={Log~\thetcbcounter: #2},
%     #1
% }

\lstdefinelanguage{json}{
    basicstyle=\ttfamily\small,
    showstringspaces=false,
    breaklines=true,
    frame=single,
    string=[s]{"}{"},
    comment=[l]{//},
    morecomment=[s]{/*}{*/},
    literate=
     *{0}{{{\color{blue}0}}}{1}
      {1}{{{\color{blue}1}}}{1}
      {2}{{{\color{blue}2}}}{1}
      {3}{{{\color{blue}3}}}{1}
      {4}{{{\color{blue}4}}}{1}
      {5}{{{\color{blue}5}}}{1}
      {6}{{{\color{blue}6}}}{1}
      {7}{{{\color{blue}7}}}{1}
      {8}{{{\color{blue}8}}}{1}
      {9}{{{\color{blue}9}}}{1}
}