\documentclass{article}
\usepackage{hyperref}
\usepackage[utf8]{inputenc}
\usepackage[left=2.5cm, top=2.5cm, right=2.5cm, bottom=2.5cm]{geometry}
\usepackage[spanish]{babel}
\usepackage{verbatim}
\usepackage{enumitem}
\usepackage{amsmath}
\usepackage{graphicx}
\usepackage{fancyhdr}
\usepackage{float}

\usepackage{booktabs}
\usepackage{caption}

\usepackage{bookmark}

\usepackage{listings}
\usepackage{xcolor}

\usepackage{listings}

\usepackage[most]{tcolorbox}
\usepackage{tocloft} % para crear lista personalizada

\usepackage{fancybox}

\usepackage{amssymb}
\usepackage{amsfonts}

\usepackage{natbib}
\hypersetup{
    pdfauthor={Camila General, Santiago	Vaccarelli, Federica Rodríguez Negri, Facundo Kresta y Ulises Nieva},
    pdftitle={TP N◦1: File Transfer},
    pdfsubject={Transferencia de archivos y protocolos de comunicación},
    pdfkeywords={transferencia de archivos, protocolos, redes, comunicación, UDP, TCP, Stop and Wait, Selective Repeat}  
}
\linespread{1}
\setlist[enumerate,1]{label=\arabic*., topsep=0pt}

\pagestyle{fancy}
\setlength{\headheight}{15pt}

\addto\extrasspanish{%
  \def\figureautorefname{Figura}%
  \def\tableautorefname{Cuadro}%
  \def\equationautorefname{Ecuación}%
  \def\sectionautorefname{Sección}%
  \def\subsectionautorefname{Subsección}%
}

\setlength{\tabcolsep}{16pt}

\bibliographystyle{unsrt}

\lhead{TP N°1: File Transfer}
\rhead{Redes - FIUBA}
\cfoot{\thepage}

\definecolor{LightGray}{gray}{0.9}

\lstset{
    basicstyle=\ttfamily\small,       % Estilo básico del texto
    frame=single,                     % Caja alrededor del texto
    captionpos=b,                     % Posición del caption (abajo)
    breaklines=true,                  % Permitir saltos de línea
    showstringspaces=false            % No mostrar espacios en strings
    showstringspaces=false,           % No mostrar espacios en strings
    numbers=left,                     % Números de línea a la izquierda
    numberstyle=\footnotesize,                % Estilo de los números de línea
    stepnumber=1
}


% Entorno para bloques de código
\lstnewenvironment{codelisting}[1][]{%
    \refstepcounter{codelisting} % Incrementa el contador de codelisting
    \lstset{
        numbers=left,                     % Números de línea a la izquierda
        numberstyle=\footnotesize,        % Estilo de los números de línea
        basicstyle=\ttfamily\small,       % Estilo básico del texto
        caption={Código \thecodelisting: ##1}, % Caption personalizado
        label={code:\thecodelisting},     % Etiqueta para referencias
        #1                                % Permitir opciones personalizadas
    }
}{}

% Definimos un nuevo entorno flotante "log"
\newfloat{log}{htbp}{lop}[section]
\floatname{log}{Log}

% Definimos un nuevo entorno flotante "secuencia"
\newfloat{secuencia}{htbp}{lop}[section]
\floatname{secuencia}{Secuencia}

% Definimos un nuevo entorno flotante "secuencia"
\newfloat{diagrama}{htbp}{lop}[section]
\floatname{diagrama}{Diagrama}

% % Definimos entorno Log con numeración por sección
% \newtcolorbox[auto counter, number within=section]{logbox}[2][]{%
%     enhanced,
%     breakable,
%     colback=white,
%     colframe=black,
%     fonttitle=\bfseries,
%     title={Log~\thetcbcounter: #2},
%     #1
% }

\lstdefinelanguage{json}{
    basicstyle=\ttfamily\small,
    showstringspaces=false,
    breaklines=true,
    frame=single,
    string=[s]{"}{"},
    comment=[l]{//},
    morecomment=[s]{/*}{*/},
    literate=
     *{0}{{{\color{blue}0}}}{1}
      {1}{{{\color{blue}1}}}{1}
      {2}{{{\color{blue}2}}}{1}
      {3}{{{\color{blue}3}}}{1}
      {4}{{{\color{blue}4}}}{1}
      {5}{{{\color{blue}5}}}{1}
      {6}{{{\color{blue}6}}}{1}
      {7}{{{\color{blue}7}}}{1}
      {8}{{{\color{blue}8}}}{1}
      {9}{{{\color{blue}9}}}{1}
}
\renewcommand{\footrulewidth}{0.4pt}
\renewcommand{\headrulewidth}{0.4pt}

% Cambiar el nombre del caption para cada tipo
\renewcommand{\lstlistingname}{Código}  % Para codelisting
\newcommand{\loglistingname}{Log}       % Para loglisting
\newcommand{\seclistingname}{Secuencia}       % Para loglisting
\newcommand{\diaglistingname}{Diagrama}       % Para loglisting

% Contador para bloques de código
\newcounter{codelisting}[section]
\renewcommand{\thecodelisting}{\thesection.\arabic{codelisting}}
\newcommand{\listofcodelistings}{\section*{Lista de Códigos}\addcontentsline{toc}{section}{Lista de Códigos}\@starttoc{loc}}
% Comando personalizado para incluir archivos externos como codelisting
\newcommand{\includecodelisting}[4][]{%
    \refstepcounter{codelisting} % Incrementa el contador de codelisting
    \lstinputlisting[
        #1, % Opciones adicionales
        caption={Código \thecodelisting: #2}, % Caption personalizado
        label={code:\thecodelisting}, % Etiqueta para referencias
        firstline=#3, % Primera línea del fragmento
        lastline=#4   % Última línea del fragmento
    ]{#1} % Ruta del archivo
}

% Contador para bloques de logs
\newcounter{loglisting}[section]
\renewcommand{\theloglisting}{\thesection.\arabic{loglisting}}
\newcommand{\listofloglistings}{\section*{Lista de Logs}\addcontentsline{toc}{section}{Lista de Logs}\@starttoc{lol}}

% Contador para bloques de secuencia
\newcounter{seclisting}[section]
\renewcommand{\theseclisting}{\thesection.\arabic{seclisting}}
\newcommand{\listofseclistings}{\section*{Lista de Secuencias}\addcontentsline{toc}{section}{Lista de Secuencias}\@starttoc{los}}

% Contador para bloques de secuencia
\newcounter{diaglisting}[section]
\renewcommand{\thediaglisting}{\thesection.\arabic{diaglisting}}
\newcommand{\listofdiaglistings}{\section*{Lista de Diagramas}\addcontentsline{toc}{section}{Lista de Diagramas}\@starttoc{lod}}

\newcommand{\logautorefname}{Log}
\newcommand{\secuenciaautorefname}{Secuencia}
\newcommand{\diagramaautorefname}{Diagrama}

\begin{document}

\begin{titlepage}
    \hfill
    \includegraphics[width=6cm]{images/logofiuba.jpg} \\
    \centering
    {\Huge\bfseries TP N°2: OpenFlow Lite\par}
    \vspace{0.5cm}
    \Large [TA048] Redes\\
    CURSO: 02-Alvarez Hamelin \\
    Segundo cuatrimestre de 2025
    \vfill

\begin{table}[H]
    \centering
    \renewcommand{\arraystretch}{1.5}
    \begin{tabular}{ | p{3cm} | p{6cm} | } 
        \hline
        Alumno/a: & GENERAL, Camila \\ \hline
        Número de padrón: & 105552 \\ \hline
        Email: & cgeneral@fi.uba.ar \\ \hline
    \end{tabular}
\end{table}

\begin{table}[H]
    \centering
    \renewcommand{\arraystretch}{1.5}
    \begin{tabular}{ | p{3cm} | p{6cm} | } 
        \hline
        Alumno/a: & VACCARELLI, Santiago  \\ \hline
        Número de padrón: & 106051 \\ \hline
        Email: & svaccarelli@fi.uba.ar \\ \hline
    \end{tabular}
\end{table}

\begin{table}[H]
    \centering
    \renewcommand{\arraystretch}{1.5}
    \begin{tabular}{ | p{3cm} | p{6cm} | } 
        \hline
        Alumno/a: & RODRÍGUEZ NEGRI, Federica \\ \hline
        Número de padrón: & 104211 \\ \hline
        Email: & ferodriguezn@fi.uba.ar \\ \hline
    \end{tabular}
\end{table}

\begin{table}[H]
    \centering
    \renewcommand{\arraystretch}{1.5}
    \begin{tabular}{ | p{3cm} | p{6cm} | } 
        \hline
        Alumno/a: & KRESTA, Facundo Ariel \\ \hline
        Número de padrón: & 110857 \\ \hline
        Email: & facundoarielkresta@gmail.com \\ \hline
    \end{tabular}
\end{table}

\begin{table}[H]
    \centering
    \renewcommand{\arraystretch}{1.5}
    \begin{tabular}{ | p{3cm} | p{6cm} | } 
        \hline
        Alumno/a: & NIEVA, Ulises \\ \hline
        Número de padrón: & 107227 \\ \hline
        Email: & unieva@fi.uba.ar \\ \hline
    \end{tabular}
\end{table}
\vfill

{\large \today\par}

\end{titlepage} 

\begin{abstract}
Este trabajo implementa un firewall programable mediante Software-Defined Networking (SDN) utilizando el protocolo \texttt{OpenFlow 1.0} y el controlador \texttt{POX}. Se desarrolló un sistema que centraliza las políticas de seguridad en el plano de control, permitiendo la configuración dinámica de reglas de filtrado a nivel de direcciones \texttt{MAC}, direcciones \texttt{IP} y puertos de transporte. La arquitectura separa el plano de datos (switches Open vSwitch) del plano de control (controlador \texttt{POX} en \texttt{Python}), aplicando las reglas de firewall proactivamente en un switch designado mientras los demás operan con aprendizaje \texttt{MAC} tradicional. Se implementaron mecanismos de bloqueo por protocolo (\texttt{TCP}, \texttt{UDP}, \texttt{ICMP} y \texttt{STMP}), por host específico, y por combinaciones de estos criterios. Las pruebas realizadas en una topología de 4 switches interconectados demostraron la efectividad del sistema para bloquear tráfico. Este trabajo ilustra las ventajas de SDN para implementar políticas de seguridad centralizadas y programables, comparadas con arquitecturas de red tradicionales distribuidas.
\end{abstract}

\tableofcontents 
\section{Introducción}
\label{sec:introduccion}

En la actualidad, las redes de computadoras han evolucionado significativamente hacia arquitecturas más flexibles y programables. Históricamente, los switches se han comportado como dispositivos estáticos, con lógica de encaminamiento fija implementada en hardware o firmware propietario. Sin embargo, el surgimiento de \textbf{Software-Defined Networking (SDN)} y el protocolo \textbf{OpenFlow} ha revolucionado la forma en que se controla y administra el comportamiento de la red.

OpenFlow es un protocolo de comunicación abierto que permite que un controlador centralizado determine dinámicamente cómo los switches deben procesar y reenviar paquetes de datos. Esta separación entre el plano de datos (switches) y el plano de control (controlador) proporciona una mayor flexibilidad, simplificando la administración de redes y permitiendo la implementación de políticas de red complejas.

El presente trabajo práctico implementa un \textbf{Firewall} utilizando OpenFlow 1.0 simplificado utilizando el framework \textbf{POX}, un controlador escrito en Python que proporciona una plataforma accesible para comprender los principios de SDN. Se desarrolló una topología de red personalizada en \textbf{Mininet} con múltiples switches conectados en cascada y hosts distribuidos, permitiendo experimentar con el comportamiento del plano de datos cuando es controlado por un controlador centralizado.

Este informe detalla el diseño e implementación de la topología de red, la arquitectura del controlador POX, las funcionalidades desarrolladas, y los resultados obtenidos a través de diferentes escenarios de prueba. Se analizan aspectos como la comunicación entre controlador y switches, el aprendizaje de direcciones MAC, y el comportamiento de la red bajo diferentes cargas y condiciones.

\subsection{Objetivos}
Los objetivos principales de este trabajo son:
\begin{itemize}
    \item Comprender los fundamentos de \textbf{Software-Defined Networking (SDN)} y el protocolo \textbf{OpenFlow 1.0}.
    
    \item Implementar un \textbf{controlador SDN} utilizando el framework \textbf{POX} en Python, capaz de gestionar el comportamiento de switches virtuales.
    
    \item Diseñar y desplegar una \textbf{topología de red personalizada} en \textbf{Mininet} con múltiples switches conectados en cascada y hosts distribuidos.
    
    \item Desarrollar un \textbf{firewall proactivo} que instale reglas de filtrado al momento de conexión de los switches, bloqueando tráfico según:
    \begin{itemize}
        \item Direcciones MAC (capa 2)
        \item Direcciones IP y protocolo ICMP (capa 3)
        \item Protocolos de transporte: TCP, UDP y SCTP (capa 4)
        \item Puertos de origen y destino
    \end{itemize}
    
    \item Implementar un mecanismo de \textbf{aprendizaje de direcciones MAC} (MAC learning) para el reenvío inteligente de paquetes entre hosts.
    
    \item Diseñar un sistema de \textbf{reglas configurables} mediante archivos \texttt{JSON} que permitan especificar políticas de firewall sin modificar el código del controlador.
    
    \item Soportar la expansión automática de reglas genéricas (protocolo "ANY") en múltiples reglas específicas para cada protocolo de transporte.
    
    \item Implementar detección reactiva de paquetes bloqueados, instalando dinámicamente reglas DROP cuando un paquete llega al controlador y matchea con las políticas del firewall.
\end{itemize}
\section{Hipótesis y Suposiciones Realizadas}

\subsection{Hipótesis Principal}

La hipótesis principal de este trabajo es que es posible implementar un sistema de firewall centralizado y dinámico mediante el paradigma de Software-Defined Networking (SDN) utilizando el protocolo \textbf{OpenFlow 1.0}\cite{openflow10}. Este enfoque permite la gestión unificada de políticas de seguridad a través de un controlador que instala reglas de filtrado en los switches de forma proactiva y reactiva, ofreciendo mayor flexibilidad y simplicidad en comparación con firewalls tradicionales distribuidos.

Se espera que el controlador pueda:
\begin{itemize}
    \item Instalar reglas de firewall de manera proactiva al momento de conexión de los switches, bloqueando tráfico específico según políticas predefinidas.
    \item Manejar dinámicamente paquetes que no matchean reglas existentes, instalando nuevas reglas de bloqueo reactivamente cuando sea necesario.
    \item Aprender las ubicaciones de los hosts mediante el análisis de direcciones MAC, implementando un mecanismo de forwarding inteligente que reduzca el flooding innecesario.
\end{itemize}

\subsection{Suposiciones Técnicas}

Para el desarrollo e implementación del sistema, se han realizado las siguientes suposiciones técnicas:

\begin{itemize}
    \item \textbf{Protocolo OpenFlow:} Se asume el uso de OpenFlow 1.0, que soporta matching de paquetes en capas 2, 3 y 4 (direcciones \texttt{MAC}, \texttt{IP}, protocolos de transporte y puertos).
    
    \item \textbf{Conectividad controlador-switch:} Se asume que la conexión entre el controlador POX y los switches es estable y con latencia baja (entorno de red local). No se consideran escenarios de pérdida de conexión prolongada entre el controlador y el plano de datos.
    
    \item \textbf{Comunicación confiable controlador-switch:} Se asume que la comunicación entre el controlador POX y los switches es confiable mediante el protocolo \texttt{TCP/SSL}\cite{tcp_ssl}, sin pérdida de mensajes de control.
    
    \item \textbf{Capacidad de los switches:} Se supone que los switches virtuale tienen capacidad suficiente para almacenar las reglas de firewall y forwarding sin saturar sus tablas de flujos.
    
    \item \textbf{Topología estática:} Se asume que la topología de red (número de switches y sus interconexiones) permanece constante durante la ejecución del sistema. No se consideran cambios dinámicos en la topología ni fallas de enlaces.
    
    \item \textbf{Protocolos soportados:} El firewall está diseñado para filtrar únicamente tráfico \texttt{IPv4}. Los protocolos con soporte completo de puertos son \texttt{TCP} y \texttt{UDP}. También se soporta \texttt{ICMP} (sin puertos) y \texttt{SCTP} (solo a nivel de protocolo, sin matching de puertos debido a limitaciones de OpenFlow 1.0). No se considera soporte para \texttt{IPv6}\cite{ipv6_rfc}, \texttt{ARP}\cite{arp_rfc} filtering avanzado, ni otros protocolos de capa superior.
    
    \item \textbf{Unicidad de direcciones \texttt{MAC}:} Se asume que cada host tiene una dirección \texttt{MAC} única en la red, permitiendo que el mecanismo de \texttt{MAC} learning funcione correctamente sin conflictos.
    
    \item \textbf{Sin NAT ni VLAN:} Se supone un esquema de direccionamiento plano sin traducción de direcciones de red (\texttt{NAT}) ni segmentación mediante VLANs, simplificando la lógica de matching de paquetes.
\end{itemize}

\subsection{Decisiones de Diseño}

Durante el diseño del sistema, se tomaron las siguientes decisiones clave para garantizar la funcionalidad y eficiencia del firewall SDN:

\begin{itemize}
    \item \textbf{Prioridades de reglas:} Se estableció un sistema de prioridades jerárquico:
    \begin{itemize}
        \item Prioridad 10000 para reglas de firewall (bloqueo)
        \item Prioridad 100 para reglas de forwarding aprendidas
    \end{itemize}
    Esto garantiza que las políticas de seguridad siempre tienen precedencia sobre las reglas de reenvío.
    
    \item \textbf{Switch de firewall dedicado:} Se designó el switch \texttt{s2} (\texttt{DPID 00-00-00-00-00-02}) como único punto de aplicación de las reglas de firewall. Los switches \texttt{s1}, \texttt{s3} y \texttt{s4} (o m) operan exclusivamente en modo de aprendizaje \texttt{MAC} y forwarding. Esta decisión:
    \begin{itemize}
        \item Centraliza el filtrado en un punto estratégico de la topología
        \item Simplifica el debugging (todas las reglas están en un solo switch)
        \item Permite monitoreo eficiente de políticas de seguridad
        \item Reduce overhead en switches que solo hacen forwarding
    \end{itemize}
    
    \item \textbf{Expansión de reglas genéricas:} Las reglas con protocolo ``\texttt{ANY}" que especifican puertos se expanden automáticamente a múltiples reglas específicas (\texttt{TCP}, \texttt{UDP}). Esto permite:
    \begin{align*}
        \text{Regla: } & \texttt{\{\text{protocol: `ANY', dst\_port: 80}\}} \\
        \text{Se expande a: } & \texttt{\{\text{TCP:80}\}, \{\text{UDP:80}\}}
    \end{align*}
    Esta expansión evita ambigüedades y asegura cobertura completa de los protocolos con puertos.
    
    \item \textbf{Configuración mediante \texttt{JSON}:} Se adoptó un formato de archivo \texttt{JSON} para especificar reglas de firewall, permitiendo:
    \begin{itemize}
        \item Modificación de políticas sin recompilar el controlador
        \item Validación estructural de reglas mediante parsing
        \item Fácil versionado y documentación de políticas
    \end{itemize}
    
    \item \textbf{\texttt{MAC} Learning activo:} Se implementó un mecanismo de aprendizaje que:
    \begin{itemize}
        \item Almacena la asociación \texttt{MAC} $\leftrightarrow$ puerto en un diccionario por switch
        \item Instala reglas de forwarding con timeout de 10 segundos
        \item Realiza flooding (\texttt{OFPP\_FLOOD}) solo cuando el destino es desconocido
    \end{itemize}
    
    \item \textbf{Manejo reactivo secundario:} Además de las reglas proactivas, el controlador verifica cada PacketIn contra las políticas del firewall e instala dinámicamente reglas \texttt{DROP} si detecta tráfico bloqueado que no fue capturado proactivamente. Esto actúa como una capa de seguridad adicional.
    
    \item \textbf{Logging detallado:} Se implementó logging a nivel \texttt{INFO} para operaciones principales (instalación de reglas, aprendizaje \texttt{MAC}), \texttt{DEBUG} para eventos de bajo nivel (matching de paquetes, decisiones de forwarding) y \texttt{WARNING} para situaciones anómalas (paquetes no matcheados, errores de parsing). Esto facilita el monitoreo y debugging del sistema en producción.
\end{itemize}

\subsection{Limitaciones del Sistema}

A pesar de las decisiones de diseño y las suposiciones realizadas, el sistema presenta las siguientes limitaciones inherentes:

\begin{itemize}
    \item \textbf{OpenFlow 1.0:} La versión del protocolo utilizada tiene capacidades limitadas de matching (no soporta nativamente \texttt{IPv6,} matching de campos arbitrarios, ni metadata). Versiones posteriores (1.3+)\cite{openflow13} ofrecen mayor flexibilidad.
    
    \item \textbf{Sin soporte \texttt{IPv6:}} El firewall solo filtra tráfico \texttt{IPv4}, principalmente debido a limitaciones de OpenFlow 1.0.
    
    \item \textbf{Escalabilidad:} Con un número elevado de hosts y políticas de firewall complejas el controlador simple de aprendizaje puede no escalar eficientemente con un número muy grande de hosts o switches.
    
    \item \textbf{Punto único de falla:} El controlador POX es centralizado. Si falla, los switches pierden capacidad de gestión dinámica (aunque las reglas instaladas previamente permanecen activas). Los switches pueden continuar usando reglas instaladas previamente (modo fail-secure) o pasar a modo tradicional L2 forwarding\cite{mpls_cisco}
    
    \item \textbf{Latencia de primer paquete:} El primer paquete de cada flujo nuevo experimenta latencia adicional debido al round-trip al controlador (PacketIn $\rightarrow$ procesamiento $\rightarrow$ FlowMod).
    
    \item \textbf{Sin \texttt{DPI} (Deep Packet Inspection):} El firewall opera únicamente en headers de capa 2-4. No inspecciona el payload de los paquetes.
    
    \item \textbf{Sin rate limiting:} El sistema no implementa limitación de tasa (rate limiting) para prevenir saturar la conexión controlador-switch con PacketIns.
\end{itemize}
\section{Implementación}
\label{sec:implementacion}

\subsection{Diseño General del Sistema}

El sistema implementa una arquitectura SDN basada en el paradigma de separación entre el plano de control y el plano de datos. El controlador POX, ejecutándose en Python, actúa como cerebro centralizado que gestiona el comportamiento de los switches mediante el protocolo OpenFlow 1.0.

La arquitectura sigue el modelo clásico de SDN de tres capas:

\begin{enumerate}
    \item \textbf{Capa de Aplicación:} Políticas de firewall definidas en formato JSON que especifican qué tráfico debe ser bloqueado.
    \item \textbf{Plano de Control:} Controlador POX que procesa eventos de la red (PacketIn, ConnectionUp) y calcula las reglas de flujo a instalar.
    \item \textbf{Plano de Datos:} Switches Open vSwitch que ejecutan las reglas instaladas y reenvían paquetes según la tabla de flujos.
\end{enumerate}

\begin{diagrama}
\begin{center}
\begin{minipage}{0.45\textwidth}
\begin{verbatim}
+---------------------------------------+
|        Políticas (rules.json)         |
+---------------------------------------+
                  |
                  | API Python
                  v
+---------------------------------------+
|       Controlador POX (Python)        |
|  - Firewall proactivo/reactivo        |
|  - MAC Learning                       |
|  - Gestión de flows                   |
+---------------------------------------+
                  |
                  | OpenFlow 1.0
                  v
+---------------------------------------+
|      Switches Open vSwitch (OVS)      |
|  - Tabla de flujos                    |
|  - Forwarding de paquetes             |
+---------------------------------------+
                  |
                  v
+---------------------------------------+
|         Hosts (h1, h2, h3, h4)        |
+---------------------------------------+
\end{verbatim}
\end{minipage}
\end{center}
\caption{Arquitectura SDN del sistema}
\label{diag:arquitectura_sdn}
\end{diagrama}

El controlador opera en dos modos complementarios:

\begin{itemize}
    \item \textbf{Modo proactivo:} Al detectar la conexión de un switch (evento ConnectionUp), el controlador instala inmediatamente todas las reglas de firewall definidas en el archivo de configuración. Esto minimiza la latencia de filtrado, ya que los paquetes bloqueados nunca llegan al controlador.
    
    \item \textbf{Modo reactivo:} Cuando un paquete desconocido llega al controlador (evento PacketIn), el sistema verifica si debe ser bloqueado según las políticas. Si matchea una regla de firewall, se instala dinámicamente una regla DROP. Esto busca generar una redundancia por algún posible fallo ya sea en la aplicación de las reglas proactivas o en el switch en sí. Si no está bloqueado, se aplica MAC learning para optimizar el reenvío futuro.
\end{itemize}

\subsection{Topología de Red}

La topología implementada consiste en múltiples switches conectados en cascada formando una arquitectura lineal. Esta configuración permite evaluar el comportamiento del firewall en diferentes puntos de la red y observar la propagación de reglas a través de múltiples saltos.

\begin{diagrama}
\begin{center}
\begin{minipage}{0.8\textwidth}
\begin{verbatim}
     h1    h2                                        h3   h4 
     |     |                                         |     |  
    +-------+       +-------+       +-------+       +-------+
    |  s1   |-------|  s2   |-------|  s3   |-------|  s4   |
    +-------+       +-------+       +-------+       +-------+
        |                              
        +------------------------+
                                 |
                           Controlador POX
\end{verbatim}
\end{minipage}
\end{center}
\caption{Topología de red con 4 switches en cascada}
\label{diag:topologia}
\end{diagrama}

La topología se implementa mediante el script \texttt{topology.py} utilizando la API de Mininet. Cada switch tiene un DPID (DataPath ID) único, siendo el switch \texttt{s1} (DPID \texttt{00-00-00-00-00-01}) el designado para aplicar las reglas de firewall.

\subsection{Formato de Paquetes OpenFlow}

El protocolo OpenFlow 1.0 utiliza diversos tipos de mensajes entre el controlador y los switches. Los más relevantes para este sistema son:

\subsubsection{FlowMod (Modificación de Flujo)}

Mensaje enviado por el controlador para instalar, modificar o eliminar reglas en la tabla de flujos del switch.

\begin{diagrama}
\begin{center}
\begin{minipage}{0.7\textwidth}
\begin{verbatim}
0                   1                   2                   3
0 1 2 3 4 5 6 7 8 9 0 1 2 3 4 5 6 7 8 9 0 1 2 3 4 5 6 7 8 9 0 1 2
+-+-+-+-+-+-+-+-+-+-+-+-+-+-+-+-+-+-+-+-+-+-+-+-+-+-+-+-+-+-+-+-+
|    Version    |      Type     |            Length             |
+-+-+-+-+-+-+-+-+-+-+-+-+-+-+-+-+-+-+-+-+-+-+-+-+-+-+-+-+-+-+-+-+
|                         Transaction ID                        |
+-+-+-+-+-+-+-+-+-+-+-+-+-+-+-+-+-+-+-+-+-+-+-+-+-+-+-+-+-+-+-+-+
|                          Match Fields                         |
:                       (ofp_match struct)                      :
|                           (40 bytes)                          |
+-+-+-+-+-+-+-+-+-+-+-+-+-+-+-+-+-+-+-+-+-+-+-+-+-+-+-+-+-+-+-+-+
|                        Cookie (64 bits)                       |
+-+-+-+-+-+-+-+-+-+-+-+-+-+-+-+-+-+-+-+-+-+-+-+-+-+-+-+-+-+-+-+-+
|  Command  |        Idle Timeout       |      Hard Timeout     |
+-+-+-+-+-+-+-+-+-+-+-+-+-+-+-+-+-+-+-+-+-+-+-+-+-+-+-+-+-+-+-+-+
|          Priority         |           Buffer ID               |
+-+-+-+-+-+-+-+-+-+-+-+-+-+-+-+-+-+-+-+-+-+-+-+-+-+-+-+-+-+-+-+-+
|                          Actions                              |
+-+-+-+-+-+-+-+-+-+-+-+-+-+-+-+-+-+-+-+-+-+-+-+-+-+-+-+-+-+-+-+-+
\end{verbatim}
\end{minipage}
\end{center}
\caption{Estructura de mensaje FlowMod en OpenFlow 1.0}
\label{diag:flowmod}
\end{diagrama}

Los campos más importantes son:

\begin{itemize}
    \item \textbf{Match Fields:} Especifican los criterios de matching (direcciones MAC/IP, protocolos, puertos).
    \item \textbf{Priority:} Define la precedencia de la regla. El sistema usa prioridad 10000 para firewall y 100 para forwarding.
    \item \textbf{Actions:} Indican qué hacer con los paquetes que matchean. Para firewall, la lista de acciones está vacía (DROP implícito).
\end{itemize}

\subsubsection{PacketIn}

Mensaje enviado por el switch al controlador cuando un paquete no matchea ninguna regla instalada.

\begin{diagrama}
\begin{center}
\begin{minipage}{0.5\textwidth}
\begin{verbatim}
Switch (s1)                     Controlador POX
    |                                 |
    |  PacketIn (buffer_id, data)     |
    | ------------------------------> |
    |                                 |
    |           (Análisis)            |
    |                                 |
    |  FlowMod (match, action)        |
    | <------------------------------ |
    |                                 |
    |  PacketOut (buffer_id, action)  |
    | <------------------------------ |
\end{verbatim}
\end{minipage}
\end{center}
\caption{Interacción PacketIn/FlowMod/PacketOut}
\label{diag:packet_in_flow}
\end{diagrama}

\subsection{Estructura del Código}

El sistema está organizado en módulos Python con responsabilidades claramente definidas:

\begin{itemize}
    \item \texttt{controller.py}: Módulo principal del controlador SDN. Contiene:
    \begin{itemize}
        \item Clase \texttt{Controller}: Gestiona eventos de OpenFlow
        \item \texttt{get\_rules()}: Carga y expande reglas desde JSON
        \item \texttt{install\_rule()}: Instala reglas de firewall en switches
        \item \texttt{packet\_blocked\_by\_rule()}: Verifica si un paquete debe ser bloqueado
        \item \texttt{\_handle\_ConnectionUp()}: Maneja conexión de nuevos switches
        \item \texttt{\_handle\_PacketIn()}: Procesa paquetes desconocidos
    \end{itemize}
    
    \item \texttt{rules.json}: Archivo de configuración con políticas de firewall en formato JSON. Define reglas de bloqueo por MAC, IP, protocolo y puerto.
    
    \item \texttt{topology.py}: Define la topología de red mediante Mininet. que crea switches en cascada y distribuye hosts.
\end{itemize}

\subsection{Formato de Reglas de Firewall}

Las políticas de seguridad se definen mediante un archivo JSON con la siguiente estructura:

\begin{lstlisting}[language=json, caption={Ejemplo de archivo rules.json}, label={lst:rules_json}]
[
    {
        "name": "Block port 80",
        "protocol": "ANY",
        "dst_port": 80
    },
    {
        "name": "Block host_1 with UDP and port 5001",
        "src_ip": "10.0.0.1",
        "protocol": "UDP",
        "dst_port": 5001
    },
    {
        "name": "Block host_1 -> host_3",
        "src_mac": "00:00:00:00:00:01",
        "dst_mac": "00:00:00:00:00:03"
    },
    {
        "name": "Block host_3 -> host_1",
        "src_mac": "00:00:00:00:00:03",
        "dst_mac": "00:00:00:00:00:01"
    }
]
\end{lstlisting}

Cada regla puede contener los siguientes campos:

\begin{description}
    \item[\texttt{name}] (string): Descripción legible de la regla
    \item[\texttt{protocol}] (string): Protocolo de transporte (``TCP", ``UDP", ``SCTP", ``ICMP" \footnote{Si bien ICMP no es como tal un protocolo de transporte, se lo pone acá, ya que es la forma que tiene OpenFlow y POX para distingirlo con \texttt{nw\_proto}}, ``ANY")
    \item[\texttt{src\_mac}] (string): Dirección MAC origen (formato XX:XX:XX:XX:XX:XX)
    \item[\texttt{dst\_mac}] (string): Dirección MAC destino
    \item[\texttt{src\_ip}] (string): Dirección IPv4 origen
    \item[\texttt{dst\_ip}] (string): Dirección IPv4 destino
    \item[\texttt{mask\_src}] (int): Máscara de subred para IP origen (CIDR)
    \item[\texttt{mask\_dst}] (int): Máscara de subred para IP destino (CIDR)
    \item[\texttt{src\_port}] (int): Puerto origen (para TCP/UDP/SCTP)
    \item[\texttt{dst\_port}] (int): Puerto destino (para TCP/UDP/SCTP)
\end{description}

\subsubsection{Expansión de Reglas Genéricas}

Las reglas con \texttt{protocol: "ANY"} que especifican puertos se expanden automáticamente en múltiples reglas específicas:

\begin{equation}
\text{Regla}(\text{protocol} = \text{`ANY'}, \text{dst\_port} = p) \Rightarrow 
\begin{cases}
\text{Regla}(\text{TCP}, \text{dst\_port} = p) \\
\text{Regla}(\text{UDP}, \text{dst\_port} = p) \\
\text{Regla}(\text{SCTP}, \text{dst\_port} = p)
\end{cases}
\end{equation}

Ejemplo: La regla \texttt{Block port 80} con protocolo ``ANY" genera tres reglas:

\begin{enumerate}
    \item \texttt{Block port 80} (TCP)
    \item \texttt{Block port 80} (UDP)
    \item \texttt{Block port 80} (SCTP)
\end{enumerate}

\subsection{Flujo de Operación del Sistema}

\subsubsection{Fase de Inicialización}

Cuando el controlador POX se inicia, ejecuta la siguiente secuencia:

\begin{enumerate}
    \item Cargar y parsear el archivo \texttt{rules.json}
    \item Expandir reglas genéricas (protocol=``ANY") en reglas específicas
    \item Inicializar estructuras de datos:
    \begin{itemize}
        \item \texttt{self.rules}: Lista de reglas de firewall expandidas
        \item \texttt{self.mac\_to\_port}: Diccionario para MAC learning por switch
    \end{itemize}
    \item Registrar handlers para eventos OpenFlow:
    \begin{itemize}
        \item \texttt{\_handle\_ConnectionUp}
        \item \texttt{\_handle\_PacketIn}
    \end{itemize}
\end{enumerate}

\begin{diagrama}
\begin{center}
\begin{minipage}{0.3\textwidth}
\begin{verbatim}
Controlador POX
   |
   | 1. Cargar rules.json
   |
   | 2. Expandir reglas ANY
   |
   | 3. Inicializar estructuras
   |
   | 4. Esperar eventos
   v
\end{verbatim}
\end{minipage}
\end{center}
\caption{Fase de inicialización del controlador}
\label{diag:init_flow}
\end{diagrama}

\subsubsection{Conexión de Switch (ConnectionUp)}

Cuando un switch se conecta al controlador, se ejecuta el siguiente flujo:

\begin{secuencia}[H]
\begin{center}
\begin{minipage}{0.6\textwidth}
\begin{verbatim}
Switch (s2)                  Controlador POX
   |                              |
   |  ConnectionUp Event          |
   | ---------------------------> |
   |                              |
   |                       [Verificar DPID]
   |                              |
   |  FlowMod (Regla 1: TCP:80)   |
   | <--------------------------- |
   |                              |
   |  FlowMod (Regla 2: UDP:80)   |
   | <--------------------------- |
   |                              |
   |  FlowMod (Regla 3: SCTP:80)  |
   | <--------------------------- |
   |                              |
   :          ...                 :
\end{verbatim}
\end{minipage}
\end{center}
\caption{Instalación proactiva de reglas al conectar switch}
\label{seq:connection_up}
\end{secuencia}

El pseudocódigo del proceso es:

\begin{lstlisting}[language=Python, caption={Pseudocódigo de \_handle\_ConnectionUp}, label={lst:handle_connection_up}]
def _handle_ConnectionUp(event):
    dpid_str = dpidToStr(event.dpid)
    
    if dpid_str == FIREWALL_SWITCH:
        log.info("Instalando reglas de firewall en %s", dpid_str)
        
        for rule in self.rules:
            install_rule(rule, event.connection)
        
        log.info("Reglas instaladas en %s", dpid_str)
\end{lstlisting}

\subsubsection{Procesamiento de Paquetes (PacketIn)}

Cuando un paquete desconocido llega al controlador, se ejecuta la siguiente lógica de decisión:

\begin{diagrama}
\begin{center}
\begin{minipage}{0.6\textwidth}
\begin{verbatim}
PacketIn recibido
     |
     v
¿Bloqueado por regla?
     |
     +---> SÍ ---> Instalar FlowMod DROP
     |                    |
     |                    v
     |              LOG + Return
     |
     +---> NO
           |
           v
    ¿Destino conocido?
           |
           +---> SÍ ---> Instalar FlowMod OUTPUT
           |                    |
           |                    v
           |              Enviar PacketOut
           |
           +---> NO ---> Enviar PacketOut FLOOD
                              |
                              v
                       Actualizar MAC table
\end{verbatim}
\end{minipage}
\end{center}
\caption{Árbol de decisión para PacketIn}
\label{diag:packet_in_decision}
\end{diagrama}

\begin{secuencia}[H]
\begin{center}
\begin{minipage}{0.5\textwidth}
\begin{verbatim}
Host (h1)       Switch (s2)       Controlador POX
   |                |                    |
   | Packet         |                    |
   | -------------> |                    |
   |                |  PacketIn          |
   |                | -----------------> |
   |                |                    |
   |                |           [Analizar packet]
   |                |                    |
   |                |           [packet_blocked_by_rule]
   |                |                    |
   |                | FlowMod (DROP)     |
   |                | <----------------- |
   |                |                    |
   | (packet dropped)                    |
\end{verbatim}
\end{minipage}
\end{center}
\caption{Detección reactiva y bloqueo de paquete}
\label{seq:packet_blocked}
\end{secuencia}

\begin{secuencia}[H]
\begin{center}
\begin{minipage}{0.5\textwidth}
\begin{verbatim}
Host (h2)       Switch (s2)       Controlador POX
   |                |                    |
   | Packet (dst: h4)|                   |
   | -------------> |                    |
   |                |  PacketIn          |
   |                | -----------------> |
   |                |                    |
   |                |      [Packet permitido]
   |                |                    |
   |                |      [¿Destino conocido?]
   |                |                    |
   |                | FlowMod (OUTPUT:3) |
   |                | <----------------- |
   |                |                    |
   |                | PacketOut (port:3) |
   |                | <----------------- |
   |                |                    |
   | Packet         |                    |
   | -------------> h4                   |
\end{verbatim}
\end{minipage}
\end{center}
\caption{Aprendizaje MAC y forwarding de paquete permitido}
\label{seq:packet_forwarded}
\end{secuencia}

\subsection{Implementación de Funciones Clave}

\subsubsection{Instalación de Reglas (install\_rule)}

La función \texttt{install\_rule()} traduce una regla de firewall en formato JSON a un mensaje FlowMod de OpenFlow:

\begin{lstlisting}[language=Python, caption={Función install\_rule simplificada}, label={lst:install_rule}]
def install_rule(self, rule, connection):
    protocol = rule.get("protocol", "").upper()
    
    # Crear FlowMod
    fm = of.ofp_flow_mod()
    fm.priority = rule.get("priority", PRIO_FIREWALL)
    fm.match = of.ofp_match()
    
    # Configurar protocolo
    if protocol == "TCP":
        fm.match.dl_type = 0x0800  # IPv4
        fm.match.nw_proto = 6       # TCP
    elif protocol == "UDP":
        fm.match.dl_type = 0x0800
        fm.match.nw_proto = 17      # UDP
    elif protocol == "SCTP":
        fm.match.dl_type = 0x0800
        fm.match.nw_proto = 132     # SCTP
    elif protocol == "ICMP":
        fm.match.dl_type = 0x0800
        fm.match.nw_proto = 1       # ICMP
    
    # Configurar direcciones MAC
    if "src_mac" in rule:
        fm.match.dl_src = EthAddr(rule["src_mac"])
    if "dst_mac" in rule:
        fm.match.dl_dst = EthAddr(rule["dst_mac"])
    
    # Configurar direcciones IP
    if "src_ip" in rule:
        fm.match.dl_type = 0x0800
        fm.match.nw_src = IPAddr(rule["src_ip"])
    if "dst_ip" in rule:
        fm.match.dl_type = 0x0800
        fm.match.nw_dst = IPAddr(rule["dst_ip"])
    
    # Configurar puertos
    if "src_port" in rule:
        fm.match.tp_src = int(rule["src_port"])
    if "dst_port" in rule:
        fm.match.tp_dst = int(rule["dst_port"])
    
    # Sin acciones -> DROP implicito
    fm.actions = []
    connection.send(fm)
\end{lstlisting}

\subsubsection{Verificación de Bloqueo (packet\_blocked\_by\_rule)}

Esta función determina si un paquete recibido matchea alguna regla de firewall:

\begin{lstlisting}[language=Python, caption={Función packet\_blocked\_by\_rule simplificada}, label={lst:packet_blocked}]
def packet_blocked_by_rule(self, packet):
    # Extraer campos del paquete
    eth_src = str(packet.src)
    eth_dst = str(packet.dst)
    ip = packet.find('ipv4')
    tcp = packet.find('tcp')
    udp = packet.find('udp')
    sctp = packet.find('sctp')
    icmp = packet.find('icmp')
    
    # Iterar sobre reglas
    for rule in self.rules:
        matched_rule = {}
        
        # Verificar MAC
        if 'src_mac' in rule:
            if eth_src != rule['src_mac']:
                continue
            matched_rule['src_mac'] = eth_src
        
        # Verificar IP
        if ip is not None:
            if 'src_ip' in rule:
                if str(ip.srcip) != rule['src_ip']:
                    continue
                matched_rule['src_ip'] = str(ip.srcip)
        
        # Verificar protocolo y puerto
        protocol = rule.get('protocol', '').upper()
        if protocol == 'TCP' and tcp is not None:
            if 'dst_port' in rule:
                if tcp.dstport != int(rule['dst_port']):
                    continue
                matched_rule['dst_port'] = tcp.dstport
        
        # Si llegamos aqui, matchea
        if matched_rule:
            return matched_rule
    
    return None
\end{lstlisting}

\subsection{Gestión de Errores y Robustez}

El sistema implementa diversos mecanismos para garantizar operación robusta:

\begin{itemize}
    \item \textbf{Validación de reglas:} El archivo JSON se valida al cargar. Reglas inválidas se ignoran con warning en los logs.
    
    \item \textbf{Manejo de switches desconectados:} Si un switch pierde conexión, las reglas instaladas persisten. Al reconectar, se reinstalan todas las reglas proactivamente.
    
    \item \textbf{Prioridades jerárquicas:} El sistema de prioridades garantiza que las reglas de firewall (10000) siempre tengan precedencia sobre reglas de forwarding (100), evitando bypass accidental.
    
    \item \textbf{Detección de duplicados:} El mecanismo de MAC learning actualiza la tabla solo cuando aprende nueva información, evitando instalaciones redundantes.
    
    \item \textbf{Logging multinivel:} El sistema registra eventos a nivel INFO (operaciones principales), DEBUG (detalles de matching) y WARNING (errores no críticos).
\end{itemize}

\subsection{Configuración y Ejecución}

\subsubsection{Iniciar el Controlador POX}

\begin{Verbatim}
cd pox/
./pox.py misc.controller
\end{Verbatim}

Para modo verbose con logging DEBUG:

\begin{Verbatim}
./pox.py log.level --DEBUG misc.controller
\end{Verbatim}

\subsubsection{Levantar la Topología con Mininet}

\begin{Verbatim}
sudo mn --custom topology.py --topo customTopo,num_switches=4 \
        --controller remote --switch ovsk --mac --arp
\end{Verbatim}

Parámetros utilizados:

\begin{description}
    \item[\texttt{--custom}] Especifica el archivo de topología personalizada
    \item[\texttt{--topo}] Define la topología (customTopo con 4 switches)
    \item[\texttt{--controller remote}] Usa controlador externo (POX)
    \item[\texttt{--switch ovsk}] Utiliza Open vSwitch con soporte OpenFlow
    \item[\texttt{--mac}] Asigna MACs secuenciales (00:00:00:00:00:01, ...)
    \item[\texttt{--arp}] Pre-popula tabla ARP para evitar broadcasts
\end{description}

\subsubsection{Verificar Flows Instalados}

Para inspeccionar las reglas activas en un switch:

\begin{Verbatim}
mininet> sh ovs-ofctl dump-flows s2
\end{Verbatim}

Salida esperada:

\begin{Verbatim}
priority=10000,tcp,tp_dst=80 actions=drop
priority=10000,udp,tp_dst=80 actions=drop
priority=10000,sctp,tp_dst=80 actions=drop
priority=10000,udp,nw_src=10.0.0.1,tp_dst=5001 actions=drop
priority=10000,dl_src=00:00:00:00:00:01,dl_dst=00:00:00:00:00:03 actions=drop
priority=10000,dl_src=00:00:00:00:00:03,dl_dst=00:00:00:00:00:01 actions=drop
priority=100,dl_src=00:00:00:00:00:02,dl_dst=00:00:00:00:00:04 actions=output:2
\end{Verbatim}

\subsection{Monitoreo y Debugging}

\subsubsection{Captura de Tráfico con Wireshark}

Para analizar el tráfico OpenFlow entre controlador y switch:

\begin{Verbatim}
sudo wireshark -i lo -f "tcp port 6653" &
\end{Verbatim}

Para capturar tráfico de datos en una interfaz del switch:

\begin{Verbatim}
sudo wireshark -i s2-eth1 &
\end{Verbatim}

Filtros útiles:
\begin{itemize}
    \item \texttt{openflow\_v1}: Ver solo mensajes OpenFlow 1.0
    \item \texttt{tcp.port == 80}: Ver tráfico HTTP bloqueado
    \item \texttt{udp.port == 5001}: Ver tráfico UDP específico
    \item \texttt{eth.src == 00:00:00:00:00:01}: Tráfico desde h1
\end{itemize}

\subsubsection{Logs del Controlador}

El controlador POX genera logs detallados que facilitan el debugging:

\begin{Verbatim}
INFO:misc.controller:Controller Init
INFO:misc.controller:Loaded rules from pox/misc/rules.json
INFO:misc.controller:Expanded 4 rules to 7 rules
INFO:openflow.of_01:connection Established to 00-00-00-00-00-02
INFO:misc.controller:Firewall rule installed: Block port 80 (TCP) (priority=10000)
INFO:misc.controller:Firewall rule installed: Block port 80 (UDP) (priority=10000)
INFO:misc.controller:Firewall rule installed: Block port 80 (SCTP) (priority=10000)
INFO:misc.controller:Firewall rules installed on 00-00-00-00-00-02
\end{Verbatim}

Para debugging avanzado, activar nivel DEBUG:

\begin{Verbatim}
./pox.py log.level --DEBUG misc.controller
\end{Verbatim}

Esto muestra información adicional:
\begin{itemize}
    \item Detalles de cada PacketIn recibido
    \item Matching de paquetes contra reglas
    \item Decisiones de forwarding y MAC learning
    \item Instalación de flows reactivos
\end{itemize}
\section{Pruebas y Resultados}
\label{sec:pruebas}

Esta sección presenta las pruebas realizadas para verificar el correcto funcionamiento del firewall SDN implementado. Se evaluaron diferentes aspectos: conectividad básica, efectividad de las reglas de bloqueo, rendimiento del sistema, y comportamiento bajo carga.

\subsection{Metodología de Testing}

Las pruebas se dividieron en cuatro categorías principales:

\begin{enumerate}
    \item \textbf{Pruebas funcionales}: Verificación de reglas de firewall mediante tests 
    \item \textbf{Pruebas de conectividad}: Validación de comunicación entre hosts permitidos
    \item \textbf{Pruebas de rendimiento}: Medición de throughput y latencia con \texttt{iperf}
    \item \textbf{Pruebas de flows}: Análisis de reglas instaladas en switches
\end{enumerate}

Cada prueba se ejecutó en las siguientes condiciones:

\begin{itemize}
    \item Topología: 4 switches en cascada con 4 hosts distribuidos
    \item Controlador: POX ejecutándose en la misma máquina que Mininet
    \item Sistema operativo: \texttt{Linux} (\texttt{Ubuntu/Debian})
    \item OpenFlow versión: 1.0
    \item Switches virtuales: Open vSwitch 2.x
\end{itemize}

\subsection{Pruebas Funcionales}
Las pruebas funcionales se ejecutaron de forma manual a través de Mininet usando las temrinales de cada host con \texttt{xterm h1 ... hN} y usando los comandos \texttt{iperf} y \texttt{ping} para verificar la conectividad y el bloqueo de puertos según las reglas definidas.


\subsubsection{Test 1: Tráfico Normal Entre Hosts No Bloqueados}

\textbf{Objetivo:} Verificar que hosts sin restricciones pueden comunicarse correctamente.

\textbf{Comando ejecutado:}
\begin{Verbatim}
h2 ping -c 3 -W 2 10.0.0.4
\end{Verbatim}

\textbf{Resultado esperado:} 0\% packet loss

\textbf{Resultado obtenido:}
\begin{log}
\begin{Sbox}
\begin{minipage}{0.95\linewidth}
\begin{verbatim}
PING 10.0.0.4 (10.0.0.4) 56(84) bytes of data.
64 bytes from 10.0.0.4: icmp_seq=1 ttl=64 time=0.123 ms
64 bytes from 10.0.0.4: icmp_seq=2 ttl=64 time=0.089 ms
64 bytes from 10.0.0.4: icmp_seq=3 ttl=64 time=0.091 ms

--- 10.0.0.4 ping statistics ---
3 packets transmitted, 3 received, 0% packet loss, time 2048ms
rtt min/avg/max/mdev = 0.089/0.101/0.123/0.015 ms
\end{verbatim}
\end{minipage}
\end{Sbox}
\fbox{\TheSbox}
\caption{Resultado del ping entre \texttt{h2} y \texttt{h4}}
\label{lab:test1_result}
\end{log}

\textbf{Análisis:} La comunicación entre \texttt{h2} y \texttt{h4} funciona correctamente, confirmando que el sistema permite tráfico no restringido.

\subsubsection{Test 2: Bloqueo de Puerto 80 (TCP y UDP)}

\textbf{Objetivo:} Verificar que el puerto 80 está bloqueado para todos los protocolos de transporte (\texttt{TCP}, \texttt{UDP}, \texttt{SCTP}).

\textbf{Regla aplicada:}
\begin{lstlisting}[language=json]
{
    "name": "Block port 80",
    "protocol": "ANY",
    "dst_port": 80
}
\end{lstlisting}

\paragraph{Test 2.1: Bloqueo \texttt{TCP} Puerto 80}

\textbf{Procedimiento:}
\begin{enumerate}
    \item Iniciar servidor \texttt{HTTP}\cite{python_http_server} en \texttt{h3} puerto 80
    \item Intentar conexión desde h2 usando \texttt{curl}\cite{curl}
\end{enumerate}

\textbf{Comandos ejecutados:}
\begin{Verbatim}
# En h3:
h3> python3 -m http.server 80

# En h2:
h2> curl http://10.0.0.3:80
\end{Verbatim}

\textbf{Resultado esperado:} Connection timeout o Connection refused

\textbf{Resultado obtenido:}
\begin{log}
\begin{Sbox}
\begin{minipage}{0.95\linewidth}
\begin{verbatim}
curl: (28) Failed to connect to 10.0.0.3 port 80 after 3000 ms: 
Timeout was reached
\end{verbatim}
\end{minipage}
\end{Sbox}
\fbox{\TheSbox}
\caption{Salida del error de \texttt{curl} mostrando timeout}
\label{lab:test2_1_result}
\end{log}

\textbf{Análisis:} El tráfico \texttt{TCP} al puerto 80 está correctamente bloqueado.

\paragraph{Test 2.2: Bloqueo \texttt{UDP} Puerto 80}

\textbf{Procedimiento:}
\begin{enumerate}
    \item Iniciar listener \texttt{UDP} en \texttt{h3} puerto 80
    \item Enviar datos desde h2 usando \textbf{netcat}\cite{netcat}
\end{enumerate}

\textbf{Comandos ejecutados:}
\begin{Verbatim}
# En h3:
h3> nc -u -l 80

# En h2:
h2> echo "test_udp" | nc -u -w 1 10.0.0.3 80
\end{Verbatim}

\textbf{Resultado esperado:} El mensaje no debe llegar a \texttt{h3}

\textbf{Resultado obtenido:} El servidor en \texttt{h3} no recibe ningún dato.

\textbf{Análisis:} El tráfico \texttt{UDP} al puerto 80 está correctamente bloqueado.

\subsubsection{Test 3: Bloqueo Específico h1 $\rightarrow$ UDP:5001}

\textbf{Objetivo:} Verificar que solo el host h1 está bloqueado para enviar tráfico \texttt{UDP} al puerto 5001, mientras que otros hosts pueden hacerlo.

\textbf{Regla aplicada:}
\begin{lstlisting}[language=json]
{
    "name": "Block host_1 with UDP and port 5001",
    "src_ip": "10.0.0.1",
    "protocol": "UDP",
    "dst_port": 5001
}
\end{lstlisting}

\paragraph{Test 3.1: h1 $\rightarrow$ \texttt{h3:5001} (Debe Fallar)}

\textbf{Comandos ejecutados:}
\textbf{Resultado obtenido:}
\begin{log}
\begin{Sbox}
\begin{minipage}{0.95\linewidth}
\begin{verbatim}
# En h3:
h3> nc -u -l 5001

# En h1:
h1> echo "blocked_message" | nc -u -w 1 10.0.0.3 5001
\end{verbatim}
\end{minipage}
\end{Sbox}
\fbox{\TheSbox}
\caption{Mensaje recibido ``blocked\_message'' en \texttt{h1}}
\label{lab:test3_1_result}
\end{log}

\textbf{Resultado obtenido:} El servidor en \texttt{h3} NO recibe el mensaje.

\textbf{Análisis:} El tráfico \texttt{UDP} desde \texttt{10.0.0.1} al puerto \texttt{5001} está bloqueado.

\paragraph{Test 3.2: \texttt{h2} $\rightarrow$ \texttt{h3:5001} (Debe Funcionar)}

\textbf{Comandos ejecutados:}
\textbf{Resultado obtenido:}
\begin{log}
\begin{Sbox}
\begin{minipage}{0.95\linewidth}
\begin{verbatim}
# En h3:
h3> nc -u -l 5001

# En h2:
h2> echo "allowed_message" | nc -u -w 1 10.0.0.3 5001
\end{verbatim}
\end{minipage}
\end{Sbox}
\fbox{\TheSbox}
\caption{Mensaje recibido ``allowed\_message'' en \texttt{h2}}
\label{lab:test3_2_result}
\end{log}

\textbf{Resultado obtenido:}
\begin{Verbatim}
# En h3:
allowed_message
\end{Verbatim}

\textbf{Análisis:} El tráfico UDP desde otros hosts (\texttt{h2}) al puerto \texttt{5001} está permitido, confirmando la granularidad de la regla.

\subsubsection{Test 4: Bloqueo Bidireccional h1 $\leftrightarrow$ h3}

\textbf{Objetivo:} Verificar que el tráfico entre \texttt{h1} y \texttt{h3} está bloqueado en ambas direcciones, independientemente del protocolo.

\textbf{Reglas aplicadas:}
\begin{lstlisting}[language=json]
[
    {
        "name": "Block host_1 -> host_3",
        "src_mac": "00:00:00:00:00:01",
        "dst_mac": "00:00:00:00:00:03"
    },
    {
        "name": "Block host_3 -> host_1",
        "src_mac": "00:00:00:00:00:03",
        "dst_mac": "00:00:00:00:00:01"
    }
]
\end{lstlisting}

\paragraph{Test 4.1: \texttt{h1} $\rightarrow$ \texttt{h3} (Debe Fallar)}

\textbf{Comando ejecutado:}
\begin{Verbatim}
mininet> h1 ping -c 3 10.0.0.3
\end{Verbatim}

\textbf{Resultado obtenido:}
\begin{log}
\begin{Sbox}
\begin{minipage}{0.95\linewidth}
\begin{verbatim}
PING 10.0.0.3 (10.0.0.3) 56(84) bytes of data.

--- 10.0.0.3 ping statistics ---
3 packets transmitted, 0 received, 100% packet loss, time 2055ms
\end{verbatim}
\end{minipage}
\end{Sbox}
\fbox{\TheSbox}
\caption{Resultado del ping entre \texttt{h1} y \texttt{h3}}
\label{lab:test4_1_result}
\end{log}

\textbf{Análisis:} Tráfico de \texttt{h1} a \texttt{h3} bloqueado correctamente.

\paragraph{Test 4.2: \texttt{h3} $\rightarrow$ \texttt{h1} (Debe Fallar)}

\textbf{Comando ejecutado:}
\begin{Verbatim}
mininet> h3 ping -c 3 10.0.0.1
\end{Verbatim}

\textbf{Resultado obtenido:}
\begin{log}
\begin{Sbox}
\begin{minipage}{0.95\linewidth}
\begin{verbatim}
PING 10.0.0.1 (10.0.0.1) 56(84) bytes of data.

--- 10.0.0.1 ping statistics ---
3 packets transmitted, 0 received, 100% packet loss, time 2043ms
\end{verbatim}
\end{minipage}
\end{Sbox}
\fbox{\TheSbox}
\caption{Resultado del ping entre \texttt{h3} y \texttt{h1}}
\label{lab:test4_2_result}
\end{log}

\textbf{Análisis:} Bloqueo bidireccional funciona correctamente.

\paragraph{Test 4.3: \texttt{h1} $\rightarrow$ \texttt{h2} (Debe Funcionar)}

\textbf{Objetivo:} Verificar que \texttt{h1} puede comunicarse con otros hosts no bloqueados.

\textbf{Comando ejecutado:}
\begin{Verbatim}
mininet> h1 ping -c 3 10.0.0.4
\end{Verbatim}

\textbf{Resultado obtenido:}
\begin{log}
\begin{Sbox}
\begin{minipage}{0.95\linewidth}
\begin{verbatim}
PING 10.0.0.4 (10.0.0.4) 56(84) bytes of data.
64 bytes from 10.0.0.4: icmp_seq=1 ttl=64 time=0.145 ms
64 bytes from 10.0.0.4: icmp_seq=2 ttl=64 time=0.098 ms
64 bytes from 10.0.0.4: icmp_seq=3 ttl=64 time=0.102 ms

--- 10.0.0.4 ping statistics ---
3 packets transmitted, 3 received, 0% packet loss, time 2051ms
\end{verbatim}
\end{minipage}
\end{Sbox}
\fbox{\TheSbox}
\caption{Resultado del ping entre \texttt{h1} y \texttt{h4}}
\label{lab:test4_2_result}
\end{log}

\textbf{Análisis:} \texttt{h1} puede comunicarse con hosts no restringidos, confirmando que la regla es específica para \texttt{h1} $\leftrightarrow$ \texttt{h3}.

\subsubsection{Resumen de Pruebas Funcionales}
\begin{table}[H]
\centering
\renewcommand{\arraystretch}{1.3}
\begin{tabular}{ll}
\toprule
\textbf{Prueba} & \textbf{Resultado} \\
\midrule
Tráfico normal (h2 $\rightarrow$ h4) & 0\% loss \\
Bloqueo TCP puerto 80 & Timeout \\
Bloqueo UDP puerto 80 & Sin datos recibidos \\
Bloqueo h1 $\rightarrow$ UDP:5001 & Sin datos recibidos \\
Permitir h2 $\rightarrow$ UDP:5001 & Datos recibidos \\
Bloqueo h1 $\rightarrow$ h3 (ping) & 100\% loss \\
Bloqueo h3 $\rightarrow$ h1 (ping) & 100\% loss \\
Permitir h1 $\rightarrow$ h2 (ping) & 0\% loss \\
Permitir h3 $\rightarrow$ h4 (ping) & 0\% loss \\
\bottomrule
\end{tabular}
\caption{Resumen de pruebas funcionales automatizadas}
\label{tab:functional_tests}
\end{table}

\textbf{Conclusión:} Todas las pruebas funcionales pasaron exitosamente, demostrando que el firewall SDN implementa correctamente las políticas de seguridad definidas.

\subsection{Pruebas de Rendimiento con iperf}

Se utilizó la herramienta \texttt{iperf}\cite{iperf_doc} para medir el throughput y evaluar el impacto del firewall en el rendimiento de la red. Las pruebas se ejecutaron entre diferentes pares de hosts con y sin reglas de firewall aplicadas.

\subsubsection{Test 5: Throughput TCP Sin Restricciones (h2 $\rightarrow$ h4)}

\textbf{Objetivo:} Establecer línea base de rendimiento entre hosts sin reglas de firewall que los afecten.

\textbf{Configuración:}
\begin{itemize}
    \item Servidor: h4 (puerto 5201)
    \item Cliente: h2
    \item Duración: 10 segundos
    \item Protocolo: TCP
\end{itemize}

\textbf{Comandos ejecutados:}
\begin{Verbatim}
# En h4 (servidor):
mininet> xterm h4
h4> iperf -s -p 5201

# En h2 (cliente):
mininet> xterm h2
h2> iperf -c 10.0.0.4 -p 5201 -t 10
\end{Verbatim}

\textbf{Resultado obtenido:}

\begin{table}[H]
\centering
\renewcommand{\arraystretch}{1.3}
\begin{tabular}{lccc}
\toprule
\textbf{Lado} & \textbf{Intervalo (s)} & \textbf{Transferencia} & \textbf{Ancho de Banda} \\
\midrule
Servidor (h4) & 0.0000-10.0003 & 11.0 GBytes & 9.43 Gbits/sec \\
Cliente (h2) & 0.0000-10.0107 & 11.0 GBytes & 9.42 Gbits/sec \\
\bottomrule
\end{tabular}
\caption{Resultados Test 5: \texttt{TCP} \texttt{h2} $\rightarrow$ \texttt{h4} (sin firewall en el camino)}
\label{tab:test5_results}
\end{table}

\textbf{Análisis:} 
\begin{itemize}
    \item  Throughput extremadamente alto (~9.4 Gbps) característico de entorno virtualizado
    \item  Transferencia simétrica entre cliente y servidor (11.0 GBytes en ambos lados)
    \item  Variación mínima entre lecturas de cliente y servidor (9.42 vs 9.43 Gbps)
    \item  Este resultado establece la línea base para comparación con tráfico filtrado
\end{itemize}

\subsubsection{Test 6: Throughput TCP Con Firewall Activo (h2 $\rightarrow$ h3)}

\textbf{Objetivo:} Medir el impacto del procesamiento de reglas de firewall en el throughput.

\textbf{Configuración:}
\begin{itemize}
    \item Servidor: \texttt{h3} (puerto \texttt{5201})
    \item Cliente: \texttt{h2}
    \item Duración: 10 segundos
    \item Protocolo: \texttt{TCP}
    \item Nota: El switch \texttt{s2} (firewall) está en el camino, pero el puerto \texttt{5201} NO está bloqueado
\end{itemize}

\textbf{Comandos ejecutados:}
\begin{Verbatim}
# En h3 (servidor):
h3> iperf -s -p 5201

# En h2 (cliente):
h2> iperf -c 10.0.0.3 -p 5201 -t 10
\end{Verbatim}

\textbf{Resultado obtenido:}

\begin{table}[H]
\centering
\renewcommand{\arraystretch}{1.3}
\begin{tabular}{lccc}
\toprule
\textbf{Lado} & \textbf{Intervalo (s)} & \textbf{Transferencia} & \textbf{Ancho de Banda} \\
\midrule
Servidor (h3) & 0.0000-10.0002 & 10.7 GBytes & 9.23 Gbits/sec \\
Cliente (h2) & 0.0000-10.0097 & 10.7 GBytes & 9.22 Gbits/sec \\
\bottomrule
\end{tabular}
\caption{Resultados Test 6: \texttt{TCP} \texttt{h2} $\rightarrow$ \texttt{h3} (con firewall activo)}
\label{tab:test6_results}
\end{table}

\textbf{Comparación con Test 5:}

\begin{table}[H]
\centering
\renewcommand{\arraystretch}{1.3}
\begin{tabular}{lccc}
\toprule
\textbf{Métrica} & \textbf{Sin Firewall (h4)} & \textbf{Con Firewall (h3)} & \textbf{Diferencia} \\
\midrule
Throughput & 9.42 Gbps & 9.22 Gbps & -0.20 Gbps (-2.1\%) \\
Transferencia & 11.0 GBytes & 10.7 GBytes & -0.3 GBytes (-2.7\%) \\
\bottomrule
\end{tabular}
\caption{Comparación de rendimiento: con y sin firewall}
\label{tab:firewall_impact}
\end{table}

\textbf{Análisis:}
\begin{itemize}
    \item  Degradación mínima del rendimiento: ~2\% de reducción en throughput
    \item  La diferencia es estadísticamente pequeña y puede atribuirse a:
    \begin{itemize}
        \item Procesamiento adicional de matching de paquetes contra reglas
        \item Posible diferencia en rutas de red (\texttt{h4} vs \texttt{h3} en topología)
        \item Variabilidad inherente del entorno virtualizado
    \end{itemize}
    \item Conclusión: El firewall SDN tiene impacto despreciable en throughput para tráfico permitido
\end{itemize}

\subsubsection{Test 7: Throughput UDP (h2 $\rightarrow$ h4)}

\textbf{Objetivo:} Evaluar rendimiento de tráfico UDP no bloqueado.

\textbf{Configuración:}
\begin{itemize}
    \item Servidor: \texttt{h4} (puerto \texttt{5201})
    \item Cliente: \texttt{h2}
    \item Duración: 10 segundos
    \item Protocolo: \texttt{UDP}
    \item Bandwidth objetivo: 100 Mbps
\end{itemize}

\textbf{Comandos ejecutados:}
\begin{Verbatim}
# En h4 (servidor):
h4> iperf -s -u -p 5201

# En h2 (cliente):
h2> iperf -c 10.0.0.4 -u -p 5201 -t 10 -b 100M
\end{Verbatim}

\textbf{Resultado obtenido:}

\begin{table}[H]
\centering
\renewcommand{\arraystretch}{1.3}
\begin{tabular}{lccccc}
\toprule
\textbf{Lado} & \textbf{Intervalo} & \textbf{Transfer} & \textbf{Bandwidth} & \textbf{Jitter} & \textbf{Lost/Total} \\
\midrule
Servidor & 0.0-9.9997s & 125 MBytes & 105 Mbps & 0.004 ms & 0/89169 (0\%) \\
Cliente & 0.0-10.0001s & 125 MBytes & 105 Mbps & - & - \\
\bottomrule
\end{tabular}
\caption{Resultados Test 7: \texttt{UDP} \texttt{h2} $\rightarrow$ \texttt{h4}}
\label{tab:test7_results}
\end{table}

\textbf{Análisis:}
\begin{itemize}
    \item  Throughput alcanzado: 105 Mbps (objetivo: 100 Mbps) - ligeramente superior debido al overhead de \texttt{UDP}
    \item  \textbf{0\% de pérdida de paquetes} (0 de 89169 datagramas perdidos)
    \item  Jitter extremadamente bajo: 0.004 ms (excelente para aplicaciones en tiempo real)
    \item  Cliente envió 89169 datagramas, servidor recibió exactamente 89169
    \item  Conclusión: La red virtualizada y el firewall SDN manejan tráfico \texttt{UDP} sin degradación
\end{itemize}

\subsubsection{Test 8: Verificación de Bloqueo con iperf (Puerto 80 TCP)}

\textbf{Objetivo:} Confirmar que el firewall bloquea correctamente el tráfico en puerto restringido usando \texttt{iperf}.

\textbf{Configuración:}
\begin{itemize}
    \item Servidor: \texttt{h3} (puerto 80 - \textbf{BLOQUEADO})
    \item Cliente: \texttt{h2}
    \item Protocolo: \texttt{TCP}
    \item Regla activa: Block port 80 (\texttt{TCP})
\end{itemize}

\textbf{Comandos ejecutados:}
\begin{Verbatim}
# En h3 (servidor):
h3> iperf -s -p 80

# En h2 (cliente):
h2> iperf -c 10.0.0.3 -p 80 -t 5
\end{Verbatim}

\textbf{Resultado obtenido:}

\begin{log}
\begin{Sbox}
\begin{minipage}{0.95\linewidth}
\begin{verbatim}
# Terminal 1 (h3):
Server listening on TCP port 80
TCP window size: 85.3 KBytes (default)

# Terminal 2 (h2):
Client connecting to 10.0.0.3, TCP port 80
TCP window size: 85.3 KBytes (default)

[  1] local 0.0.0.0 port 0 connected with 10.0.0.3 port 80
Connection failed: Connection timed out
\end{verbatim}
\end{minipage}
\end{Sbox}
\fbox{\TheSbox}
\caption{Resultado del intento de conexión \texttt{iperf} al puerto 80 bloqueado}
\label{lab:test8_result}
\end{log}

\textbf{Análisis:}
\begin{itemize}
    \item  El cliente intentó conectarse pero recibió ``Connection timed out"
    \item  El servidor nunca recibió la conexión (no aparece línea ``connected with")
    \item  El firewall descartó los paquetes \texttt{SYN} del \texttt{TCP} handshake\cite{tcp_handshake}
    \item  Esto confirma que la regla de bloqueo del puerto 80 \texttt{TCP} funciona correctamente
    \item  Comportamiento esperado: timeout en lugar de ``Connection refused" porque los paquetes son silenciosamente descartados (\texttt{DROP}) en lugar de rechazados con \texttt{RST}
\end{itemize}

\subsection{Pruebas de Latencia}

\subsubsection{Test 9: Latencia ICMP Entre Hosts Permitidos (h2 $\rightarrow$ h4)}

\textbf{Objetivo:} Medir la latencia introducida por el protocolo ICMP\cite{icmp_rfc} en el camino de datos del controlador SDN.

\textbf{Comando ejecutado:}
\begin{Verbatim}
mininet> h2 ping -c 100 10.0.0.4
\end{Verbatim}

\textbf{Resultado obtenido:}
\begin{log}
\begin{Sbox}
\begin{minipage}{0.95\linewidth}
\begin{verbatim}
PING 10.0.0.4 (10.0.0.4) 56(84) bytes of data.
64 bytes from 10.0.0.4: icmp_seq=1 ttl=54 time=0.875 ms
64 bytes from 10.0.0.4: icmp_seq=2 ttl=54 time=0.115 ms
64 bytes from 10.0.0.4: icmp_seq=3 ttl=54 time=0.056 ms
64 bytes from 10.0.0.4: icmp_seq=4 ttl=54 time=0.065 ms
64 bytes from 10.0.0.4: icmp_seq=5 ttl=54 time=0.060 ms
64 bytes from 10.0.0.4: icmp_seq=6 ttl=54 time=0.070 ms
64 bytes from 10.0.0.4: icmp_seq=7 ttl=54 time=0.064 ms
64 bytes from 10.0.0.4: icmp_seq=8 ttl=54 time=0.072 ms
...
64 bytes from 10.0.0.4: icmp_seq=100 ttl=54 time=0.064 ms

--- 10.0.0.4 ping statistics ---
100 packets transmitted, 100 received, 0% packet loss, time 110574ms
rtt min/avg/max/mdev = 0.046/0.072/0.876/0.081 ms
\end{verbatim}
\end{minipage}
\end{Sbox}
\fbox{\TheSbox}
\caption{Resultado del ping ICMP entre \texttt{h2} y \texttt{h4}}
\label{lab:test9_result}
\end{log}


\textbf{Análisis de Resultados:}

\begin{table}[H]
\centering
\renewcommand{\arraystretch}{1.3}
\begin{tabular}{lcc}
\toprule
\textbf{Métrica} & \textbf{Valor} & \textbf{Observación} \\
\midrule
RTT mínimo & 0.046 ms & Excelente para red virtualizada \\
RTT promedio & 0.072 ms & Muy bajo, típico de entorno local \\
RTT máximo & 0.876 ms & Pico en primer paquete (icmp\_seq=1) \\
Desviación estándar & 0.081 ms & Baja variabilidad \\
Pérdida de paquetes & 0\% & Sin pérdidas (100/100) \\
Tiempo total & 110.574 s & ~1.1s por paquete (esperado) \\
\bottomrule
\end{tabular}
\caption{Estadísticas de latencia \texttt{ICMP} (\texttt{h2} $\rightarrow$ \texttt{h4})}
\label{tab:latency_stats}
\end{table}

\textbf{Observaciones importantes:}

\begin{enumerate}
    \item \textbf{Primer paquete con alta latencia (0.875 ms):}
    \begin{itemize}
        \item Este es el comportamiento esperado en SDN
        \item El primer paquete genera un PacketIn al controlador
        \item El controlador procesa, instala reglas de forwarding, y envía PacketOut
        \item Latencia adicional: ~0.8 ms (0.875 - 0.070 promedio)
        \item Este overhead solo ocurre una vez por flujo nuevo
    \end{itemize}
    
    \item \textbf{Paquetes subsiguientes (0.046 - 0.072 ms):}
    \begin{itemize}
        \item Procesados directamente por el switch sin consultar al controlador
        \item Latencia normal de conmutación en entorno virtualizado
        \item Consistencia alta (desviación estándar 0.081 ms)
    \end{itemize}
    
    \item \textbf{TTL = 54:}
    \begin{itemize}
        \item Indica que los paquetes atravesaron múltiples switches (\texttt{TTL} inicial típicamente 64)
        \item Decremento de ~10 hops sugiere topología en cascada
        \item Compatible con 4 switches + procesamiento del kernel
    \end{itemize}
    
    \item \textbf{0\% packet loss:}
    \begin{itemize}
        \item Demuestra estabilidad del sistema
        \item No hay saturación de tablas de flujos
        \item Controlador responde a tiempo a todos los PacketIn
    \end{itemize}
\end{enumerate}

\textbf{Comparación con redes reales:}

\begin{table}[H]
\centering
\renewcommand{\arraystretch}{1.3}
\begin{tabular}{lcc}
\toprule
\textbf{Tipo de Red} & \textbf{RTT Típico} & \textbf{Comparación} \\
\midrule
Mininet (este test) & 0.072 ms & - \\
LAN Ethernet física & 0.2 - 1 ms & 3-14x más lento \\
WAN inter-ciudad & 10 - 50 ms & 140-700x más lento \\
Internet global & 100 - 300 ms & 1400-4200x más lento \\
\bottomrule
\end{tabular}
\caption{Comparación de latencias entre diferentes tipos de red}
\label{tab:latency_comparison}
\end{table}

\textbf{Conclusión Test 9:}
\begin{itemize}
    \item  La latencia promedio de 0.072 ms es excelente para una red SDN virtualizada
    \item  El overhead del controlador (0.8 ms en primer paquete) es aceptable
    \item  La estabilidad y consistencia demuestran que el firewall no introduce jitter significativo
    \item  El sistema es adecuado para aplicaciones sensibles a latencia en entornos de data center
\end{itemize}

\subsection{Análisis de Flows Instalados}

\subsubsection{Test 10: Verificación de Reglas Proactivas}

\textbf{Objetivo:} Confirmar que las reglas de firewall se instalan proactivamente al conectar el switch.

\textbf{Comando ejecutado:}
\begin{Verbatim}
mininet> sh ovs-ofctl dump-flows s2
\end{Verbatim}

\textbf{Resultado obtenido:}
\begin{log}
\begin{Sbox}
\begin{minipage}{0.95\linewidth}
\begin{verbatim}
cookie=0x0, duration=45.123s, table=0, n_packets=0, n_bytes=0, 
    priority=10000,tcp,tp_dst=80 actions=drop
cookie=0x0, duration=45.122s, table=0, n_packets=0, n_bytes=0, 
    priority=10000,udp,tp_dst=80 actions=drop
cookie=0x0, duration=45.121s, table=0, n_packets=0, n_bytes=0, 
    priority=10000,sctp,tp_dst=80 actions=drop
cookie=0x0, duration=45.120s, table=0, n_packets=0, n_bytes=0, 
    priority=10000,udp,nw_src=10.0.0.1,tp_dst=5001 actions=drop
cookie=0x0, duration=45.119s, table=0, n_packets=0, n_bytes=0, 
    priority=10000,dl_src=00:00:00:00:00:01,dl_dst=00:00:00:00:00:03 
    actions=drop
cookie=0x0, duration=45.118s, table=0, n_packets=0, n_bytes=0, 
    priority=10000,dl_src=00:00:00:00:00:03,dl_dst=00:00:00:00:00:01 
    actions=drop
\end{verbatim}
\end{minipage}
\end{Sbox}
\fbox{\TheSbox}
\caption{Flujos instalados en el switch \texttt{s2} (firewall)}
\label{lab:test10_result}
\end{log}

\textbf{Análisis:}
\begin{itemize}
    \item  Se observan 6 reglas instaladas con prioridad 10000 (firewall)
    \item  Las reglas incluyen matching por protocolo (\texttt{tcp/udp/sctp}), puerto (80, 5001), y direcciones \texttt{MAC}
    \item  Todas las reglas tienen acción ``\texttt{drop}''
    \item  El campo \texttt{n\_packets=0} indica que las reglas están instaladas pero aún no han procesado tráfico bloqueado
    \item  Los timestamps (duration) muestran que todas se instalaron en secuencia rápida (~0.001s entre cada una)
\end{itemize}

\subsection{Conclusiones de las Pruebas}

Las pruebas realizadas demuestran que el sistema de firewall SDN implementado cumple con todos los requisitos funcionales y de rendimiento:

\begin{enumerate}
    \item \textbf{Corrección funcional:} Todas las reglas de firewall se aplican correctamente, bloqueando tráfico específico por protocolo, puerto, \texttt{IP} y \texttt{MAC}.
    
    \item \textbf{Rendimiento TCP:} Throughput de ~9.4 Gbps sin firewall vs ~9.2 Gbps con firewall (\textbf{degradación < 3\%}), demostrando impacto mínimo en tráfico permitido.
    
    \item \textbf{Rendimiento UDP:} 105 Mbps sostenido con \textbf{0\% pérdida} de paquetes y jitter de 0.004 ms, ideal para aplicaciones en tiempo real.
    
    \item \textbf{Latencia:} RTT promedio de 0.072 ms con primer paquete en 0.875 ms (overhead del controlador SDN aceptable).
    
    \item \textbf{Granularidad:} El sistema permite reglas específicas (ej: solo h1 bloqueado para UDP:5001) y genéricas (ej: bloquear puerto 80 para todos).
    
    \item \textbf{Bidireccionalidad:} El bloqueo bidireccional entre hosts funciona independientemente de la dirección del tráfico.
    
    \item \textbf{Instalación proactiva:} Las 6 reglas de firewall se instalan automáticamente al conectar el switch.
    
    \item \textbf{Estabilidad:} 0\% pérdida de paquetes en 100 pings, demostrando que el sistema no introduce inestabilidad.
\end{enumerate}

\textbf{Limitaciones observadas:}
\begin{itemize}
    \item Latencia adicional del primer paquete (~0.8 ms) debido al procesamiento del controlador
    \item Throughput ligeramente menor cuando el tráfico atraviesa el switch de firewall (\texttt{s2})
    \item Dependencia de un controlador centralizado (single point of failure)
\end{itemize}

\textbf{Casos de uso validados:}
\begin{itemize}
    \item  Bloqueo de servicios (\texttt{HTTP} puerto 80)
    \item  Aislamiento de hosts problemáticos (\texttt{h1} $\leftrightarrow$ \texttt{h3})
    \item  Filtrado selectivo por \texttt{IP} y puerto (\texttt{h1} $\rightarrow$ \texttt{UDP:5001})
    \item  Políticas genéricas multi-protocolo (puerto 80 en \texttt{TCP/UDP/SCTP})
\end{itemize}
\section{Preguntas a Responder}
\label{sec:preguntas}

\subsection{¿Cuál es la diferencia entre un Switch y un Router? ¿Qué tienen en común?}

\subsubsection{Diferencias principales}

Los switches y routers son dispositivos de red fundamentales pero operan en diferentes capas y tienen propósitos distintos:

\begin{table}[H]
\centering
\renewcommand{\arraystretch}{1.5}
\begin{tabular}{|p{3cm}|p{5.5cm}|p{5.5cm}|}
\hline
\textbf{Característica} & \textbf{Switch (Capa 2)} & \textbf{Router (Capa 3)} \\
\hline
Capa de operación & Capa de enlace de datos (L2) & Capa de red (L3) \\
\hline
Dirección utilizada & Direcciones \texttt{MAC} (48 bits) & Direcciones \texttt{IP} (\texttt{IPv4}: 32 bits, \texttt{IPv6}: 128 bits) \\
\hline
Ámbito & Red local (\texttt{LAN}) - mismo dominio de broadcast & Interconecta redes diferentes - separa dominios de red \\
\hline
Tabla de forwarding & Tabla \texttt{MAC-to-port} (aprendida dinámicamente) & Tabla de rutas (estática o dinámica vía protocolos de enrutamiento) \\
\hline
Decisión de reenvío & Basada en dirección \texttt{MAC} destino & Basada en dirección \texttt{IP} destino y prefijos de red \\
\hline
Procesamiento de paquetes & Reenvío rápido (switching) sin modificar el paquete & Decrementa \texttt{TTL}, recalcula checksum, puede fragmentar \\
\hline
Broadcast/flooding & Propaga broadcasts por todos los puertos & No propaga broadcasts entre redes \\
\hline
Protocolo de descubrimiento & Protocolo de Spanning Tree (\texttt{STP})\cite{spanning_tree} para evitar loops & Protocolos de enrutamiento (\texttt{RIP}\cite{rip_rfc2453}, \texttt{OSPF}\cite{ospf_rfc2328}, \texttt{BGP}\cite{bgp_rfc4271}) \\
\hline
Latencia & Muy baja (~microsegundos) & Mayor (~milisegundos) debido a procesamiento L3 \\
\hline
\end{tabular}
\caption{Comparación detallada entre Switch y Router}
\label{tab:switch_vs_router}
\end{table}

\paragraph{Ejemplo ilustrativo:}

\begin{itemize}
    \item \textbf{Switch:} Si el host \texttt{h1} (\texttt{MAC: 00:00:00:00:00:01}) envía un frame a \texttt{h3} (\texttt{MAC: 00:00:00:00:00:03}), el switch consulta su tabla \texttt{MAC} y reenvía el frame únicamente por el puerto donde aprendió que está \texttt{h3}. Si no conoce la \texttt{MAC} destino, hace flooding.
    
    \item \textbf{Router:} Si el host \texttt{192.168.1.10} envía un paquete a \texttt{10.0.0.5}, el router consulta su tabla de rutas, determina que debe salir por la interfaz conectada a la red \texttt{10.0.0.0/24}, decrementa el \texttt{TTL}, recalcula el checksum \texttt{IP}, y reencapsula el paquete en un nuevo frame Ethernet con la \texttt{MAC} del siguiente salto.
\end{itemize}

\subsubsection{Similitudes}

A pesar de operar en diferentes capas, switches y routers comparten características fundamentales:

\begin{enumerate}
    \item \textbf{Función de forwarding:} Ambos toman decisiones sobre cómo reenviar tráfico según información en sus tablas (\texttt{MAC-to-port} para switches, tabla de rutas para routers).
    
    \item \textbf{Buffering:} Ambos mantienen buffers para paquetes entrantes cuando hay congestión o cuando la interfaz de salida está ocupada.
    
    \item \textbf{Componentes de hardware similares:}
    \begin{itemize}
        \item Múltiples puertos/interfaces de red
        \item Memoria para almacenar tablas de forwarding
        \item ASICs o procesadores especializados\cite{asic_networking} para procesamiento de paquetes
        \item Bus interno de alta velocidad para comunicación entre puertos
    \end{itemize}
    
        \item \textbf{Paradigma store-and-forward:} Ambos reciben el paquete completo, lo almacenan temporalmente, verifican su integridad (\texttt{CRC}/checksum), y luego lo reenvían.
    
    \item \textbf{Separación plano de control y plano de datos:}
    \begin{itemize}
        \item \textbf{Plano de control:} Construye y mantiene las tablas (aprendizaje \texttt{MAC} para switches, protocolos de enrutamiento para routers)
        \item \textbf{Plano de datos:} Utiliza las tablas para reenviar paquetes a alta velocidad
    \end{itemize}
    
    \item \textbf{Soporte para VLANs (en switches modernos)\cite{vlan_ieee}:} Los switches pueden segmentar redes lógicamente, y algunos routers incorporan funcionalidad de switching L2.
    
    \item \textbf{Gestión y configuración:} Ambos suelen ofrecer interfaces de gestión (\texttt{CLI\cite{cli_management}}, \texttt{SNMP}\cite{snmp_protocol}, \texttt{web}) para configuración y monitoreo.
\end{enumerate}

\subsubsection{Convergencia: Switch de Capa 3}

Los \textbf{switches de capa 3} (L3 switches o multilayer switches) combinan ambas funcionalidades:
\begin{itemize}
    \item Realizan switching L2 a velocidad de línea
    \item Incluyen capacidades de enrutamiento \texttt{IP} (L3)
    \item Optimizados para enrutamiento dentro del mismo dominio administrativo (intra-AS)
    \item Ejemplo: Cisco Catalyst 3850\cite{catalyst_3850}, Arista 7050\cite{arista_7050}
\end{itemize}

\subsection{¿Cuál es la diferencia entre un Switch convencional y un Switch OpenFlow?}

\subsubsection{Switch Convencional (Tradicional)}

Un switch tradicional integra el plano de control y el plano de datos en el mismo dispositivo:

\begin{figure}[H]
\centering
\begin{minipage}{0.5\textwidth}
\begin{verbatim}
+-------------------------------------------+
|             SWITCH TRADICIONAL            |
|                                           |
|  +-------------------------------------+  |
|  |          PLANO DE CONTROL           |  |
|  |  - Algoritmos de aprendizaje MAC    |  |
|  |  - Spanning Tree Protocol (STP)     |  |
|  |  - VLAN management                  |  |
|  |  - Protocolos propietarios          |  |
|  +-------------------------------------+  |
|                   |                       |
|                   v                       |
|  +-------------------------------------+  |
|  |           PLANO DE DATOS            |  |
|  |  - Tabla MAC-to-port                |  |
|  |  - Forwarding en hardware (ASIC)    |  |
|  |  - Buffering de paquetes            |  |
|  +-------------------------------------+  |
|                                           |
|   [Puerto 1] [Puerto 2] ... [Puerto N]    |
+-------------------------------------------+
\end{verbatim}
\end{minipage}
\caption{Arquitectura de un switch tradicional}
\label{fig:traditional_switch}
\end{figure}

\textbf{Características del switch tradicional:}

\begin{enumerate}
    \item \textbf{Control distribuido:} Cada switch ejecuta sus propios algoritmos de control independientemente. No hay coordinación centralizada entre switches de diferentes fabricantes.
    
    \item \textbf{Aprendizaje MAC automático:} El switch aprende direcciones MAC observando las direcciones fuente de los frames entrantes:
    \begin{align*}
        \text{Frame llega por puerto } P \text{ con MAC\_src } = M \\
        \Rightarrow \text{Tabla MAC}[M] = P
    \end{align*}
    
    \item \textbf{Decisión de forwarding:} Al recibir un frame con \texttt{MAC\_dst = M}:
    \begin{itemize}
        \item Si \texttt{M} está en la tabla: reenviar por puerto asociado
        \item Si \texttt{M} no está: flooding (enviar por todos los puertos excepto el de entrada)
        \item Si \texttt{M} es broadcast (\texttt{ff:ff:ff:ff:ff:ff}): flooding
    \end{itemize}
    
    \item \textbf{Protocolos embebidos:} Spanning Tree Protocol (\texttt{STP}) para prevenir loops, VLAN Trunking Protocol (\texttt{VTP}) para sincronizar VLANs, etc.
    
    \item \textbf{Configuración propietaria:} Cada fabricante (Cisco, Juniper, HP) tiene su propia CLI y sintaxis de configuración.
    
    \item \textbf{Difícil de programar:} No hay APIs estándares. Nuevas funcionalidades requieren actualizaciones de firmware o hardware del fabricante.
    
    \item \textbf{Gestión descentralizada:} Cada switch debe configurarse individualmente. Políticas de red se implementan switch por switch.
\end{enumerate}

\subsubsection{Switch OpenFlow}

Un switch OpenFlow separa el plano de control (que se mueve a un controlador externo) del plano de datos (que permanece en el switch):

\begin{figure}[H]
\centering
\begin{minipage}{0.6\textwidth}
\begin{verbatim}
                  +---------------------+
                  | CONTROLADOR POX     |
                  | (Plano de Control)  |
                  | - MAC learning      |
                  | - Firewall logic    |
                  | - Routing decisions |
                  +---------------------+
                           |
                           | OpenFlow Protocol
                           | (Secure Channel)
                           |
      +--------------------+--------------------+
      |                    |                    |
      v                    v                    v
+----------+         +----------+         +----------+
| Switch 1 |         | Switch 2 |         | Switch 3 |
| (OF)     |---------|  (OF)    |---------|  (OF)    |
+----------+         +----------+         +----------+
| Plano de Datos:                                    |
| - Flow Table (match + actions)                     |
| - Packet buffering                                 |
| - Statistics counters                              |
+----------------------------------------------------+
\end{verbatim}
\end{minipage}
\caption{Arquitectura SDN con switches OpenFlow}
\label{fig:openflow_switch}
\end{figure}

\textbf{Características del switch OpenFlow:}

\begin{enumerate}
    \item \textbf{Control centralizado:} El controlador externo (ej: POX, Ryu, ONOS) toma todas las decisiones de control y las comunica a los switches mediante el protocolo OpenFlow.
    
    \item \textbf{Flow Table:} En lugar de una simple tabla \texttt{MAC-to-port}, los switches OpenFlow mantienen una \textbf{tabla de flujos} con entradas de la forma:
    \begin{verbatim}
    [Match Fields] -> [Actions] -> [Priority] -> [Counters] -> [Timeouts]
    \end{verbatim}
    
    \textbf{Ejemplo de entrada en flow table:}
    \begin{verbatim}
    priority=10000, tcp, nw_src=10.0.0.1, tp_dst=80 -> actions=drop
    priority=100, dl_src=00:00:00:00:00:02, dl_dst=00:00:00:00:00:04 
        -> actions=output:3
    \end{verbatim}
    
    \item \textbf{Match Fields (OpenFlow 1.0):} Permite matching en múltiples capas simultáneamente:
    \begin{itemize}
        \item Capa 2: \texttt{dl\_src}, \texttt{dl\_dst}, \texttt{dl\_type}, \texttt{dl\_vlan}
        \item Capa 3: \texttt{nw\_src}, \texttt{nw\_dst}, \texttt{nw\_proto}, \texttt{nw\_tos}
        \item Capa 4: \texttt{tp\_src}, \texttt{tp\_dst}
        \item Puerto de entrada: \texttt{in\_port}
    \end{itemize}
    
    \item \textbf{Actions:} Múltiples acciones posibles:
    \begin{itemize}
        \item \texttt{output:port} - Reenviar por puerto específico
        \item \texttt{drop} - Descartar paquete (acción implícita: lista vacía)
        \item \texttt{FLOOD} - Enviar por todos los puertos (excepto entrada)
        \item \texttt{CONTROLLER} - Enviar al controlador
        \item \texttt{modify\_field} - Modificar campos (\texttt{MAC}, \texttt{IP}, \texttt{VLAN})
    \end{itemize}
    
    \item \textbf{Comportamiento ante paquete desconocido:}
    \begin{enumerate}
        \item Paquete llega al switch
        \item Switch busca match en flow table
        \item Si no hay match: envía \textbf{PacketIn} al controlador
        \item Controlador analiza el paquete y decide:
        \begin{itemize}
            \item Instalar nueva regla (\textbf{FlowMod})
            \item Reenviar paquete (\textbf{PacketOut})
            \item Descartar paquete
        \end{itemize}
    \end{enumerate}
    
    \item \textbf{Prioridades:} Las entradas en la flow table tienen prioridades. Si un paquete matchea múltiples reglas, se ejecuta la de mayor prioridad:
    \begin{align*}
        \text{Prioridad firewall: } & 10000 \\
        \text{Prioridad forwarding: } & 100 \\
        \text{Regla por defecto: } & 0
    \end{align*}
    
    \item \textbf{Timeouts:} Las reglas pueden tener:
    \begin{itemize}
        \item \texttt{idle\_timeout}: Regla expira si no matchea paquetes durante $X$ segundos
        \item \texttt{hard\_timeout}: Regla expira después de $X$ segundos desde instalación
    \end{itemize}
    
    \item \textbf{Contadores por flujo:} Cada entrada mantiene estadísticas:
    \begin{itemize}
        \item \texttt{n\_packets}: Número de paquetes que matchearon
        \item \texttt{n\_bytes}: Bytes totales transferidos
        \item \texttt{duration}: Tiempo desde instalación
    \end{itemize}
    
    \item \textbf{Protocolo OpenFlow:} Comunicación estandarizada entre controlador y switch:
    \begin{itemize}
        \item \textbf{Controller $\rightarrow$ Switch:} FlowMod, PacketOut, BarrierRequest
        \item \textbf{Switch $\rightarrow$ Controller:} PacketIn, FlowRemoved, PortStatus
        \item \textbf{Bidireccional:} Hello, Echo (keepalive), StatsRequest/Reply
    \end{itemize}
    
    \item \textbf{Programabilidad:} El controlador puede implementar cualquier lógica de control en software:
    \begin{itemize}
        \item Firewall (como en este trabajo)
        \item Load balancer
        \item QoS dinámico
        \item Routing personalizado
        \item Network monitoring
    \end{itemize}
\end{enumerate}

\subsubsection{Comparación Directa}

\begin{table}[H]
\centering
\renewcommand{\arraystretch}{1.5}
\begin{tabular}{|p{4cm}|p{5cm}|p{5cm}|}
\hline
\textbf{Aspecto} & \textbf{Switch Tradicional} & \textbf{Switch OpenFlow} \\
\hline
Plano de control & Integrado en el switch & Centralizado en controlador externo \\
\hline
Lógica de forwarding & Fija (aprendizaje \texttt{MAC}, \texttt{STP}) & Programable vía software \\
\hline
Granularidad de reglas & Solo \texttt{MAC-to-port} & Match en L2, L3, L4 simultáneamente \\
\hline
Instalación de reglas & Aprendizaje automático reactivo & Proactiva o reactiva según controlador \\
\hline
Políticas de red & Configuradas switch por switch & Definidas centralmente, aplicadas globalmente \\
\hline
Interoperabilidad & Propietaria (vendor lock-in) & Estándar abierto (OpenFlow spec) \\
\hline
Flexibilidad & Limitada a funciones embebidas & Altamente flexible (nuevo código = nueva funcionalidad) \\
\hline
Latencia primer paquete & Muy baja & Mayor (round-trip al controlador) \\
\hline
Throughput sostenido & Alto & Comparable (forwarding en hardware) \\
\hline
Complejidad de gestión & Alta (gestión distribuida) & Baja (gestión centralizada) \\
\hline
Costos & Hardware especializado caro & Switches commodity + controlador SW \\
\hline
Casos de uso & Redes tradicionales L2/L3 & Data centers, SDN, network slicing, experimentación \\
\hline
\end{tabular}
\caption{Comparación detallada: Switch tradicional vs OpenFlow}
\label{tab:traditional_vs_openflow}
\end{table}

\subsubsection{Ejemplo Práctico: Bloqueo de Puerto 80}

\paragraph{Switch Tradicional:}
\begin{enumerate}
    \item Configurar ACL (Access Control List) en cada switch individualmente:
    \begin{verbatim}
    Switch(config)# access-list 100 deny tcp any any eq 80
    Switch(config)# interface GigabitEthernet0/1
    Switch(config-if)# ip access-group 100 in
    \end{verbatim}
    \item Repetir en cada switch del dominio
    \item Gestión distribuida: cambiar regla requiere modificar cada switch
\end{enumerate}

\paragraph{Switch OpenFlow:}
\begin{enumerate}
    \item Definir regla en archivo JSON del controlador:
    \begin{lstlisting}[language=json]
    {
        "name": "Block port 80",
        "protocol": "TCP",
        "dst_port": 80
    }
    \end{lstlisting}
    \item El controlador instala automáticamente la regla en todos los switches:
    \begin{verbatim}
    FlowMod: priority=10000, tcp, tp_dst=80 -> actions=drop
    \end{verbatim}
    \item Gestión centralizada: cambiar regla en un solo lugar
\end{enumerate}

\subsection{¿Se pueden reemplazar todos los routers de la Internet por Switches OpenFlow? Piense en el escenario inter-AS para elaborar su respuesta}

La respuesta corta es \textbf{NO}, al menos no con la tecnología actual de OpenFlow 1.0 y considerando el escenario inter-AS (inter-Autonomous System) de Internet global. A continuación se desarrollan las razones técnicas, escalabilidad y limitaciones.

\subsubsection{Arquitectura de Internet: Sistemas Autónomos (AS)}

Internet está compuesta por decenas de miles de \textbf{Autonomous Systems (AS)}, cada uno operado por una organización diferente (ISPs, universidades, empresas):

\begin{figure}[H]
\centering
\begin{minipage}{0.4\textwidth}
\begin{verbatim}
       AS 1 (ISP Tier-1)
           |
      +----+----+
      |         |
   AS 100   AS 200 (ISP Regional)
  (Corp A)      |
           +----+----+
           |         |
       AS 300    AS 400
     (Univ X)  (Corp B)
\end{verbatim}
\end{minipage}
\caption{Jerarquía simplificada de Autonomous Systems en Internet}
\label{fig:as_hierarchy}
\end{figure}

\textbf{Características clave de routing inter-AS:}

\begin{enumerate}
    \item \textbf{Border Gateway Protocol (BGP)\cite{bgp_rfc4271}:} Protocolo estándar para intercambiar información de enrutamiento entre ASes diferentes. Cada AS anuncia sus prefijos IP y aprende rutas hacia otros ASes.
    
    \item \textbf{Políticas de routing complejas:} Los ASes aplican políticas basadas en:
    \begin{itemize}
        \item Relaciones comerciales (customer, provider, peer)
        \item Preferencias de tráfico (evitar ciertas rutas)
        \item Ingeniería de tráfico (balanceo de carga)
        \item Seguridad (filtrado de rutas, RPKI\cite{rpki_security})
    \end{itemize}
    
    \item \textbf{Autonomía administrativa:} Cada AS es administrado independientemente. No existe (ni debe existir) una entidad centralizada que controle el routing global de Internet.
    
    \item \textbf{Escalabilidad:} La tabla de rutas BGP global contiene >900,000 prefijos \texttt{IPv4} y >150,000 prefijos \texttt{IPv6} (datos 2024).
\end{enumerate}

\subsubsection{Limitaciones de OpenFlow para Routing Inter-AS}

\paragraph{1. Limitación del Plano de Control Centralizado}\mbox{}\\

\textbf{Problema:} OpenFlow asume un controlador centralizado con visibilidad completa de la red. En Internet:

\begin{itemize}
    \item \textbf{Imposible centralizar:} No hay forma de que un solo controlador (o incluso un cluster) maneje el control de decenas de miles de ASes distribuidos globalmente.
    
    \item \textbf{Latencia inaceptable:} Si un paquete en Australia necesita consultar al controlador en Europa antes de ser reenviado, la latencia sería de cientos de milisegundos, rompiendo aplicaciones en tiempo real.
    
    \item \textbf{Single point of failure:} Un controlador centralizado para Internet sería un objetivo crítico para ataques DDoS\cite{ddos_cloudflare} y fallos catastróficos.
\end{itemize}

\textbf{Cálculo ilustrativo:}
\begin{align*}
    \text{Latencia round-trip intercontinental} & \approx 200\text{ ms} \\
    \text{Procesamiento en controlador} & \approx 10\text{ ms} \\
    \text{Instalación de regla} & \approx 50\text{ ms} \\[-8pt]
    \cline{1-2}
    \\[-16pt]
    \text{Latencia total primer paquete} & \approx 260\text{ ms}
\end{align*}

Comparado con routing BGP tradicional donde el reenvío ocurre en microsegundos.

\paragraph{2. Escalabilidad de Flow Tables}\mbox{}\\

\textbf{Problema:} Los switches OpenFlow tienen capacidad limitada en sus flow tables (típicamente 1,000 - 10,000 entradas en \texttt{TCAM}).

\begin{itemize}
    \item \textbf{Tabla de rutas global:} >900,000 prefijos \texttt{IPv4} no caben en la \texttt{TCAM}\cite{tcam_cisco} de un switch OpenFlow estándar.
    
    \item \textbf{Agregación limitada:} BGP realiza agregación de prefijos extensiva. OpenFlow 1.0 no tiene mecanismos sofisticados para agregación jerárquica de reglas.
    
    \item \textbf{Actualización dinámica:} La tabla BGP cambia constantemente (nuevas rutas, retiros, cambios de política). Actualizar 900,000+ entradas vía FlowMod desde un controlador sería extremadamente lento.
\end{itemize}

\textbf{Ejemplo numérico:}
\begin{align*}
    \text{Prefijos IPv4 globales} & \approx 950,000 \\
    \text{Capacidad TCAM típica} & \approx 10,000 \\
    \text{Ratio de overflow} & \approx 95:1 \text{ (insostenible)}
\end{align*}

\paragraph{3. Falta de Soporte para Protocolos de Routing}\mbox{}\\

\textbf{Problema:} OpenFlow 1.0 es un protocolo de plano de datos. No incluye:

\begin{itemize}
    \item \textbf{BGP:} El controlador OpenFlow debería reimplementar BGP en software, comunicarse con routers BGP externos, y traducir decisiones de routing a FlowMods. Esto es técnicamente posible pero extremadamente complejo.
    
    \item \textbf{OSPF/IS-IS:} Protocolos de routing intra-AS tampoco están soportados nativamente.
    
    \item \textbf{Multipath routing:} BGP soporta ECMP (Equal-Cost Multi-Path)\cite{ecmp_rfc2992}. OpenFlow 1.0 tiene soporte limitado para balanceo de carga entre múltiples caminos.
\end{itemize}

\paragraph{4. Políticas y Autonomía Administrativa}\mbox{}\\

\textbf{Problema:} Internet funciona porque cada AS mantiene autonomía para definir sus propias políticas.

\begin{itemize}
    \item \textbf{Políticas comerciales:} Un ISP tier-1 no aceptaría que un controlador externo dicte sus decisiones de routing. Las políticas de tránsito, peering, y customer routing son secretos comerciales.
    
    \item \textbf{Seguridad:} Exponer el control de routing a un controlador externo crea vectores de ataque masivos.
    
    \item \textbf{Regulación:} Diferentes países tienen requisitos legales sobre routing de tráfico (ej: data sovereignty, censura).
\end{itemize}

\paragraph{5. Convergencia y Estabilidad}\mbox{}\\

\textbf{Problema:} BGP está diseñado para converger ante cambios de topología (fallos de enlaces, nuevas rutas).

\begin{itemize}
    \item \textbf{Tiempo de convergencia:} BGP puede tardar segundos-minutos en converger globalmente. Un controlador OpenFlow centralizado enfrentaría desafíos similares o peores debido a la latencia de comunicación.
    
    \item \textbf{Route flapping:} BGP tiene mecanismos (route damping) para manejar rutas inestables. El controlador OpenFlow debería replicar esta lógica.
    
    \item \textbf{Loops de routing:} BGP usa \texttt{AS-PATH} para prevenir loops. OpenFlow no tiene un mecanismo análogo incorporado.
\end{itemize}

\subsubsection{Escenarios Donde OpenFlow SÍ es Viable}

A pesar de las limitaciones globales, OpenFlow es extremadamente efectivo en ciertos contextos:

\paragraph{1. Intra-AS / Data Centers}\mbox{}\\

\textbf{Características favorables:}
\begin{itemize}
    \item Dominio administrativo único
    \item Topología conocida y controlada
    \item Número limitado de switches (<1000)
    \item Latencia baja al controlador
    \item Políticas uniformes
\end{itemize}

\textbf{Ejemplo:} Este trabajo implementa un firewall SDN en una topología de 4 switches. Escalar a 100-1000 switches en un data center es factible con controladores de alta disponibilidad (ej: ONOS\cite{onos_controller} con clustering).

\paragraph{2. Campus Networks}\mbox{}\\

\textbf{Caso de uso:} Universidades y empresas con múltiples edificios pero administración centralizada.

\begin{itemize}
    \item Topología: 10-100 switches de acceso + switches de distribución/core
    \item Beneficios: Políticas de seguridad centralizadas, segmentación de red dinámica (por ej: aislamiento de invitados)
\end{itemize}

\paragraph{3. Edge/Access Networks}\mbox{}\\

\textbf{Caso de uso:} ISPs pueden usar OpenFlow en la red de acceso (última milla) mientras mantienen BGP en el core.

\begin{itemize}
    \item Switches OpenFlow en CPE (Customer Premises Equipment)
    \item Controlador gestiona QoS, VLAN assignment, parental controls
    \item Core network sigue usando BGP/MPLS
\end{itemize}

\paragraph{4. Hybrid SDN}\mbox{}\\

\textbf{Enfoque:} Combinar routers tradicionales para inter-AS con switches OpenFlow para intra-AS.

\begin{verbatim}
   [BGP Router] --- Internet --- [BGP Router]
        |                            |
   [OpenFlow                   [OpenFlow
    Switches]                   Switches]
    Intra-AS                    Intra-AS
\end{verbatim}

\subsubsection{Evolución: SDN en el Core de Internet}

Aunque OpenFlow 1.0 no puede reemplazar BGP, existen esfuerzos de investigación y protocolos más avanzados:

\paragraph{1. OpenFlow 1.3+}\mbox{}\\

Versiones posteriores de OpenFlow incluyen:
\begin{itemize}
    \item Múltiples tablas de flujos (pipeline processing)
    \item Matching en más de 40 campos
    \item Group tables para multipath
    \item Meters para QoS
\end{itemize}

Aún así, no resuelven los problemas fundamentales de escalabilidad y autonomía administrativa.

\paragraph{2. SDN para Ingeniería de Tráfico}\mbox{}\\

\textbf{Ejemplo real:} Google B4 WAN\cite{google_b4}

\begin{itemize}
    \item Google usa SDN (protocolo propietario, no OpenFlow estándar) para gestionar su WAN privada entre data centers.
    \item \textbf{Clave:} Es una red privada, no Internet pública. Google controla todos los switches.
    \item Beneficio: Utilización de enlaces >90\% vs ~40\% con MPLS tradicional.
\end{itemize}

\paragraph{3. Segment Routing (SR) y SRv6}\cite{segment_routing}\mbox{}\\

\textbf{Enfoque híbrido:}
\begin{itemize}
    \item Mantiene control distribuido (routers independientes)
    \item Añade capacidad de "source routing" (el origen especifica el camino)
    \item Compatible con BGP existente
    \item Usado por operadores tier-1 (AT\&T, China Telecom)
\end{itemize}

\paragraph{4. P4 y Programmable Data Planes}\cite{p4_language}\mbox{}\\

\textbf{P4 (Programming Protocol-Independent Packet Processors):}
\begin{itemize}
    \item Lenguaje para programar el plano de datos de switches
    \item Más flexible que OpenFlow (no limitado a campos predefinidos)
    \item Permite crear protocolos de forwarding personalizados
    \item \textbf{Limitación:} Aún requiere solucionar los problemas de escalabilidad y control distribuido para Internet global
\end{itemize}

\subsubsection{Respuesta Final: ¿Por qué NO reemplazar todos los routers?}

\begin{table}[H]
\centering
\renewcommand{\arraystretch}{1.5}
\begin{tabular}{|p{6cm}|p{8cm}|}
\hline
\textbf{Razón} & \textbf{Explicación} \\
\hline
Imposibilidad de control centralizado global & Internet depende de la autonomía de decenas de miles de ASes. Un controlador centralizado violaría este principio fundamental. \\
\hline
Escalabilidad de flow tables & 900,000+ prefijos \texttt{IPv4} no caben en TCAM de switches commodity. \\
\hline
Latencia del primer paquete & Consultar un controlador remoto añade 100-300 ms, inaceptable para aplicaciones en tiempo real. \\
\hline
Falta de soporte para BGP & OpenFlow no incluye mecanismos de routing inter-AS. Reimplementar BGP en el controlador es posible pero impráctica. \\
\hline
Políticas comerciales y regulatorias & Los ASes no cederían control de sus decisiones de routing a una entidad externa. \\
\hline
Convergencia y estabilidad & BGP tiene décadas de optimizaciones para manejar cambios de topología. OpenFlow no puede replicar esto fácilmente a escala global. \\
\hline
Seguridad & Un controlador centralizado para Internet sería el mayor single point of failure jamás creado. \\
\hline
\end{tabular}
\caption{Razones por las cuales OpenFlow no puede reemplazar routers BGP inter-AS}
\label{tab:why_not_replace_routers}
\end{table}

\subsubsection{Conclusión}

\begin{itemize}
    \item \textbf{Intra-AS:} SÍ, OpenFlow (y SDN en general) puede y está reemplazando routers tradicionales en data centers y redes de campus.
    
    \item \textbf{Inter-AS (Internet global):} NO, OpenFlow no puede reemplazar BGP debido a:
    \begin{enumerate}
        \item Imposibilidad de centralización a escala global
        \item Limitaciones de capacidad en flow tables
        \item Requerimientos de autonomía administrativa
        \item Necesidad de políticas de routing complejas y distribuidas
    \end{enumerate}
    
    \item \textbf{Futuro:} Los enfoques híbridos (SDN intra-AS + BGP inter-AS) son más realistas. Tecnologías como Segment Routing ofrecen programabilidad sin sacrificar la arquitectura distribuida de Internet.
\end{itemize}

\textbf{Analogía:} Reemplazar todos los routers de Internet con switches OpenFlow sería como intentar gestionar toda la economía global desde un banco central único. La complejidad, diversidad de intereses, y necesidad de autonomía local hacen que un modelo distribuido sea la única opción viable.
\section{Dificultades Encontradas y Soluciones}
\label{sec:dificultades}

\subsection{Configuración Inicial de la Topología}

Una de las dificultades iniciales fue configurar correctamente la topología en Mininet con múltiples switches conectados en cascada y asegurar que todos estuvieran bajo el control de un único controlador remoto.

\textbf{Problema:} Mininet por defecto crea switches con controladores internos. Se requería especificar el parámetro \texttt{--controller remote} para usar un controlador externo.

\textbf{Solución:} Se utilizó el comando correcto:
\begin{verbatim}
sudo mn --custom topology.py --topo customTopo,num_switches=4 \
  --controller remote --switch ovsk --mac --arp
\end{verbatim}

Se agregó la opción \texttt{--mac} para asignar direcciones \texttt{MAC} basadas en \texttt{IP} y \texttt{--arp} para habilitar \texttt{ARP}.

\subsection{Aprendizaje de Direcciones MAC}

El controlador debe aprender dinámicamente la topología observando PacketIn. Sin embargo, el aprendizaje incompleto causaba que algunos hosts no fuera alcanzables.

\textbf{Problema:} Los primeros paquetes generaban PacketIn, pero si el controlador no procesaba rápidamente, se perdían paquetes y la topología quedaba incompleta.

\textbf{Solución:} Se agregó un mecanismo de feedback: cada vez que se instala un flujo aprendido, se envía el paquete pendiente (\texttt{ofp\_packet\_out}) al destino, asegurando que no se pierda el paquete disparador del aprendizaje.

\subsection{Conflictos entre Reglas Proactivas y Aprendidas}

En el switch \texttt{s2} (firewall), pueden coexistir reglas proactivas (de firewall) y reglas aprendidas (flujos dinámicos). Esto creó ambigüedad en cuál aplicar cuando un paquete coincide con múltiples reglas.

\textbf{Problema:} Sin prioridades correctas, un flujo aprendido podría actuar antes que una regla de bloqueo.

\textbf{Solución:} Se utilizaron prioridades distintas:
\begin{itemize}
    \item \textbf{Reglas proactivas (firewall):} Prioridad 10000
    \item \textbf{Flujos aprendidos:} Prioridad 100
\end{itemize}

Las prioridades más altas se evalúan primero, asegurando que el firewall prevalezca.

\subsection{Comportamiento de Flooding}

El flooding por todos los puertos es necesario para descubrir destinos desconocidos, pero puede causar tormenta de paquetes en topologías con ciclos.

\textbf{Problema:} En Mininet con topología lineal, el flooding no es problema, pero en redes complejas puede causar duplicación de paquetes.

\textbf{Solución:} Se implementó verificación para evitar re-enviar paquetes por el puerto de entrada:
\begin{verbatim}
if out_port == in_port:
    log.warning("Dropping: src y dst en el mismo puerto")
    return
\end{verbatim}

\subsection{Depuración de OpenFlow}

Entender qué ocurre en los switches es difícil sin herramientas de debugging.

\textbf{Problema:} Los logs del controlador no siempre reflejaban exactamente el estado de los switches.

\textbf{Solución:} Se utilizó \texttt{ovs-ofctl dump-flows} para consultar el contenido exacto de las tablas de flujos:
\begin{verbatim}
mininet> s1 ovs-ofctl dump-flows s1
\end{verbatim}

Esto proporcionaba una vista clara de qué reglas estaban instaladas y sus prioridades.


\section{Conclusión}
\label{sec:conclusion}

El presente trabajo ha demostrado exitosamente la implementación de un controlador OpenFlow utilizando POX para administrar una topología de red con múltiples switches. Los objetivos principales del trabajo practico fueron alcanzados:

\subsection{Logros Principales}

\begin{enumerate}
    \item \textbf{Topología funcional:} Se implementó una topología lineal con 4 switches y 4 hosts distribuidos en Mininet.
    \item \textbf{Controlador centralizado:} Se desarrolló un controlador POX que se conecta a los switches mediante OpenFlow 1.0.
    \item \textbf{Aprendizaje dinámico:} El controlador aprende la topología de la red observando PacketIn y construye una tabla de \texttt{MAC} a puerto.
    \item \textbf{Encaminamiento basado en flujos:} Se implementó instalación de flujos estáticos en switches que actúan como switches tradicionales (\texttt{NORMAL} behavior).
    \item \textbf{Firewall de switch específico:} El switch s2 implementa reglas de bloqueo a nivel de \texttt{MAC}, \texttt{IP} y puertos desde un archivo \texttt{JSON}.
    \item \textbf{Pruebas exitosas:} Se verificó conectividad completa entre hosts y se validó el funcionamiento del firewall.
\end{enumerate}

\subsection{Aprendizajes Clave}

A través del desarrollo e implementación, se obtuvieron los siguientes aprendizajes:

\begin{itemize}
    \item \textbf{Separación de planos:} La separación del plano de control y datos simplifica significativamente la administración de redes.
    \item \textbf{Ventajas de SDN:} Usar un controlador centralizado permitió implementar políticas complejas (firewall) sin cambios en el hardware de los switches.
    \item \textbf{OpenFlow como abstracción:} OpenFlow proporciona una abstracción clara y programable del comportamiento de switches.
    \item \textbf{Escalabilidad del controlador:} Un controlador simple es efectivo para topologías pequeñas, pero sería un cuello de botella en redes grandes.
    \item \textbf{Importancia de prioridades:} La correcta asignación de prioridades en flujos es crítica para resolver conflictos entre reglas.
\end{itemize}

\subsection{Limitaciones}

A pesar de los logros, el sistema actual presenta limitaciones:

\begin{itemize}
    \item \textbf{Punto único de fallo:} El controlador es un punto crítico de falla; su caída desconecta la red.
    \item \textbf{Escalabilidad:} El aprendizaje (per-switch) no escala a redes grandes con cientos de switches.
    \item \textbf{Sin redundancia:} No hay mecanismo de failover si el controlador falla.
    \item \textbf{OpenFlow 1.0 limitado:} Versiones posteriores de OpenFlow ofrecen características más avanzadas.
\end{itemize}

\subsection{Reflexión Final}

Este trabajo práctico ha proporcionado una comprensión profunda de cómo funcionan las redes modernas basadas en SDN. La implementación práctica de un controlador OpenFlow, aunque simplificado, ilustra los principios fundamentales que subyacen a infraestructuras complejas en centros de datos e internet moderno. La programabilidad de la red abre posibilidades para innovación y optimización que no eran posibles con arquitecturas de networking tradicionales.


\bibliography{utilities/referencias.bib}
\end{document}