\section{Pruebas y Resultados}
\label{sec:pruebas}

Esta sección presenta las pruebas realizadas para verificar el correcto funcionamiento del firewall SDN implementado. Se evaluaron diferentes aspectos: conectividad básica, efectividad de las reglas de bloqueo, rendimiento del sistema, y comportamiento bajo carga.

\subsection{Metodología de Testing}

Las pruebas se dividieron en cuatro categorías principales:

\begin{enumerate}
    \item \textbf{Pruebas funcionales}: Verificación de reglas de firewall mediante tests 
    \item \textbf{Pruebas de conectividad}: Validación de comunicación entre hosts permitidos
    \item \textbf{Pruebas de rendimiento}: Medición de throughput y latencia con \texttt{iperf}
    \item \textbf{Pruebas de flows}: Análisis de reglas instaladas en switches
\end{enumerate}

Cada prueba se ejecutó en las siguientes condiciones:

\begin{itemize}
    \item Topología: 4 switches en cascada con 4 hosts distribuidos
    \item Controlador: POX ejecutándose en la misma máquina que Mininet
    \item Sistema operativo: \texttt{Linux} (\texttt{Ubuntu/Debian})
    \item OpenFlow versión: 1.0
    \item Switches virtuales: Open vSwitch 2.x
\end{itemize}

\subsection{Pruebas Funcionales}
Las pruebas funcionales se ejecutaron de forma manual a través de Mininet usando las temrinales de cada host con \texttt{xterm h1 ... hN} y usando los comandos \texttt{iperf} y \texttt{ping} para verificar la conectividad y el bloqueo de puertos según las reglas definidas.


\subsubsection{Test 1: Tráfico Normal Entre Hosts No Bloqueados}

\textbf{Objetivo:} Verificar que hosts sin restricciones pueden comunicarse correctamente.

\textbf{Comando ejecutado:}
\begin{Verbatim}
h2 ping -c 3 -W 2 10.0.0.4
\end{Verbatim}

\textbf{Resultado esperado:} 0\% packet loss

\textbf{Resultado obtenido:}
\begin{log}
\begin{Sbox}
\begin{minipage}{0.95\linewidth}
\begin{verbatim}
PING 10.0.0.4 (10.0.0.4) 56(84) bytes of data.
64 bytes from 10.0.0.4: icmp_seq=1 ttl=64 time=0.123 ms
64 bytes from 10.0.0.4: icmp_seq=2 ttl=64 time=0.089 ms
64 bytes from 10.0.0.4: icmp_seq=3 ttl=64 time=0.091 ms

--- 10.0.0.4 ping statistics ---
3 packets transmitted, 3 received, 0% packet loss, time 2048ms
rtt min/avg/max/mdev = 0.089/0.101/0.123/0.015 ms
\end{verbatim}
\end{minipage}
\end{Sbox}
\fbox{\TheSbox}
\caption{Resultado del ping entre \texttt{h2} y \texttt{h4}}
\label{lab:test1_result}
\end{log}

\textbf{Análisis:} La comunicación entre \texttt{h2} y \texttt{h4} funciona correctamente, confirmando que el sistema permite tráfico no restringido.

\subsubsection{Test 2: Bloqueo de Puerto 80 (TCP y UDP)}

\textbf{Objetivo:} Verificar que el puerto 80 está bloqueado para todos los protocolos de transporte (\texttt{TCP}, \texttt{UDP}, \texttt{SCTP}).

\textbf{Regla aplicada:}
\begin{lstlisting}[language=json]
{
    "name": "Block port 80",
    "protocol": "ANY",
    "dst_port": 80
}
\end{lstlisting}

\paragraph{Test 2.1: Bloqueo \texttt{TCP} Puerto 80}

\textbf{Procedimiento:}
\begin{enumerate}
    \item Iniciar servidor \texttt{HTTP}\cite{python_http_server} en \texttt{h3} puerto 80
    \item Intentar conexión desde h2 usando \texttt{curl}\cite{curl}
\end{enumerate}

\textbf{Comandos ejecutados:}
\begin{Verbatim}
# En h3:
h3> python3 -m http.server 80

# En h2:
h2> curl http://10.0.0.3:80
\end{Verbatim}

\textbf{Resultado esperado:} Connection timeout o Connection refused

\textbf{Resultado obtenido:}
\begin{log}
\begin{Sbox}
\begin{minipage}{0.95\linewidth}
\begin{verbatim}
curl: (28) Failed to connect to 10.0.0.3 port 80 after 3000 ms: 
Timeout was reached
\end{verbatim}
\end{minipage}
\end{Sbox}
\fbox{\TheSbox}
\caption{Salida del error de \texttt{curl} mostrando timeout}
\label{lab:test2_1_result}
\end{log}

\textbf{Análisis:} El tráfico \texttt{TCP} al puerto 80 está correctamente bloqueado.

\paragraph{Test 2.2: Bloqueo \texttt{UDP} Puerto 80}

\textbf{Procedimiento:}
\begin{enumerate}
    \item Iniciar listener \texttt{UDP} en \texttt{h3} puerto 80
    \item Enviar datos desde h2 usando \textbf{netcat}\cite{netcat}
\end{enumerate}

\textbf{Comandos ejecutados:}
\begin{Verbatim}
# En h3:
h3> nc -u -l 80

# En h2:
h2> echo "test_udp" | nc -u -w 1 10.0.0.3 80
\end{Verbatim}

\textbf{Resultado esperado:} El mensaje no debe llegar a \texttt{h3}

\textbf{Resultado obtenido:} El servidor en \texttt{h3} no recibe ningún dato.

\textbf{Análisis:} El tráfico \texttt{UDP} al puerto 80 está correctamente bloqueado.

\subsubsection{Test 3: Bloqueo Específico h1 $\rightarrow$ UDP:5001}

\textbf{Objetivo:} Verificar que solo el host h1 está bloqueado para enviar tráfico \texttt{UDP} al puerto 5001, mientras que otros hosts pueden hacerlo.

\textbf{Regla aplicada:}
\begin{lstlisting}[language=json]
{
    "name": "Block host_1 with UDP and port 5001",
    "src_ip": "10.0.0.1",
    "protocol": "UDP",
    "dst_port": 5001
}
\end{lstlisting}

\paragraph{Test 3.1: h1 $\rightarrow$ \texttt{h3:5001} (Debe Fallar)}

\textbf{Comandos ejecutados:}
\textbf{Resultado obtenido:}
\begin{log}
\begin{Sbox}
\begin{minipage}{0.95\linewidth}
\begin{verbatim}
# En h3:
h3> nc -u -l 5001

# En h1:
h1> echo "blocked_message" | nc -u -w 1 10.0.0.3 5001
\end{verbatim}
\end{minipage}
\end{Sbox}
\fbox{\TheSbox}
\caption{Mensaje recibido ``blocked\_message'' en \texttt{h1}}
\label{lab:test3_1_result}
\end{log}

\textbf{Resultado obtenido:} El servidor en \texttt{h3} NO recibe el mensaje.

\textbf{Análisis:} El tráfico \texttt{UDP} desde \texttt{10.0.0.1} al puerto \texttt{5001} está bloqueado.

\paragraph{Test 3.2: \texttt{h2} $\rightarrow$ \texttt{h3:5001} (Debe Funcionar)}

\textbf{Comandos ejecutados:}
\textbf{Resultado obtenido:}
\begin{log}
\begin{Sbox}
\begin{minipage}{0.95\linewidth}
\begin{verbatim}
# En h3:
h3> nc -u -l 5001

# En h2:
h2> echo "allowed_message" | nc -u -w 1 10.0.0.3 5001
\end{verbatim}
\end{minipage}
\end{Sbox}
\fbox{\TheSbox}
\caption{Mensaje recibido ``allowed\_message'' en \texttt{h2}}
\label{lab:test3_2_result}
\end{log}

\textbf{Resultado obtenido:}
\begin{Verbatim}
# En h3:
allowed_message
\end{Verbatim}

\textbf{Análisis:} El tráfico UDP desde otros hosts (\texttt{h2}) al puerto \texttt{5001} está permitido, confirmando la granularidad de la regla.

\subsubsection{Test 4: Bloqueo Bidireccional h1 $\leftrightarrow$ h3}

\textbf{Objetivo:} Verificar que el tráfico entre \texttt{h1} y \texttt{h3} está bloqueado en ambas direcciones, independientemente del protocolo.

\textbf{Reglas aplicadas:}
\begin{lstlisting}[language=json]
[
    {
        "name": "Block host_1 -> host_3",
        "src_mac": "00:00:00:00:00:01",
        "dst_mac": "00:00:00:00:00:03"
    },
    {
        "name": "Block host_3 -> host_1",
        "src_mac": "00:00:00:00:00:03",
        "dst_mac": "00:00:00:00:00:01"
    }
]
\end{lstlisting}

\paragraph{Test 4.1: \texttt{h1} $\rightarrow$ \texttt{h3} (Debe Fallar)}

\textbf{Comando ejecutado:}
\begin{Verbatim}
mininet> h1 ping -c 3 10.0.0.3
\end{Verbatim}

\textbf{Resultado obtenido:}
\begin{log}
\begin{Sbox}
\begin{minipage}{0.95\linewidth}
\begin{verbatim}
PING 10.0.0.3 (10.0.0.3) 56(84) bytes of data.

--- 10.0.0.3 ping statistics ---
3 packets transmitted, 0 received, 100% packet loss, time 2055ms
\end{verbatim}
\end{minipage}
\end{Sbox}
\fbox{\TheSbox}
\caption{Resultado del ping entre \texttt{h1} y \texttt{h3}}
\label{lab:test4_1_result}
\end{log}

\textbf{Análisis:} Tráfico de \texttt{h1} a \texttt{h3} bloqueado correctamente.

\paragraph{Test 4.2: \texttt{h3} $\rightarrow$ \texttt{h1} (Debe Fallar)}

\textbf{Comando ejecutado:}
\begin{Verbatim}
mininet> h3 ping -c 3 10.0.0.1
\end{Verbatim}

\textbf{Resultado obtenido:}
\begin{log}
\begin{Sbox}
\begin{minipage}{0.95\linewidth}
\begin{verbatim}
PING 10.0.0.1 (10.0.0.1) 56(84) bytes of data.

--- 10.0.0.1 ping statistics ---
3 packets transmitted, 0 received, 100% packet loss, time 2043ms
\end{verbatim}
\end{minipage}
\end{Sbox}
\fbox{\TheSbox}
\caption{Resultado del ping entre \texttt{h3} y \texttt{h1}}
\label{lab:test4_2_result}
\end{log}

\textbf{Análisis:} Bloqueo bidireccional funciona correctamente.

\paragraph{Test 4.3: \texttt{h1} $\rightarrow$ \texttt{h2} (Debe Funcionar)}

\textbf{Objetivo:} Verificar que \texttt{h1} puede comunicarse con otros hosts no bloqueados.

\textbf{Comando ejecutado:}
\begin{Verbatim}
mininet> h1 ping -c 3 10.0.0.4
\end{Verbatim}

\textbf{Resultado obtenido:}
\begin{log}
\begin{Sbox}
\begin{minipage}{0.95\linewidth}
\begin{verbatim}
PING 10.0.0.4 (10.0.0.4) 56(84) bytes of data.
64 bytes from 10.0.0.4: icmp_seq=1 ttl=64 time=0.145 ms
64 bytes from 10.0.0.4: icmp_seq=2 ttl=64 time=0.098 ms
64 bytes from 10.0.0.4: icmp_seq=3 ttl=64 time=0.102 ms

--- 10.0.0.4 ping statistics ---
3 packets transmitted, 3 received, 0% packet loss, time 2051ms
\end{verbatim}
\end{minipage}
\end{Sbox}
\fbox{\TheSbox}
\caption{Resultado del ping entre \texttt{h1} y \texttt{h4}}
\label{lab:test4_2_result}
\end{log}

\textbf{Análisis:} \texttt{h1} puede comunicarse con hosts no restringidos, confirmando que la regla es específica para \texttt{h1} $\leftrightarrow$ \texttt{h3}.

\subsubsection{Resumen de Pruebas Funcionales}
\begin{table}[H]
\centering
\renewcommand{\arraystretch}{1.3}
\begin{tabular}{ll}
\toprule
\textbf{Prueba} & \textbf{Resultado} \\
\midrule
Tráfico normal (h2 $\rightarrow$ h4) & 0\% loss \\
Bloqueo TCP puerto 80 & Timeout \\
Bloqueo UDP puerto 80 & Sin datos recibidos \\
Bloqueo h1 $\rightarrow$ UDP:5001 & Sin datos recibidos \\
Permitir h2 $\rightarrow$ UDP:5001 & Datos recibidos \\
Bloqueo h1 $\rightarrow$ h3 (ping) & 100\% loss \\
Bloqueo h3 $\rightarrow$ h1 (ping) & 100\% loss \\
Permitir h1 $\rightarrow$ h2 (ping) & 0\% loss \\
Permitir h3 $\rightarrow$ h4 (ping) & 0\% loss \\
\bottomrule
\end{tabular}
\caption{Resumen de pruebas funcionales automatizadas}
\label{tab:functional_tests}
\end{table}

\textbf{Conclusión:} Todas las pruebas funcionales pasaron exitosamente, demostrando que el firewall SDN implementa correctamente las políticas de seguridad definidas.

\subsection{Pruebas de Rendimiento con iperf}

Se utilizó la herramienta \texttt{iperf}\cite{iperf_doc} para medir el throughput y evaluar el impacto del firewall en el rendimiento de la red. Las pruebas se ejecutaron entre diferentes pares de hosts con y sin reglas de firewall aplicadas.

\subsubsection{Test 5: Throughput TCP Sin Restricciones (h2 $\rightarrow$ h4)}

\textbf{Objetivo:} Establecer línea base de rendimiento entre hosts sin reglas de firewall que los afecten.

\textbf{Configuración:}
\begin{itemize}
    \item Servidor: h4 (puerto 5201)
    \item Cliente: h2
    \item Duración: 10 segundos
    \item Protocolo: TCP
\end{itemize}

\textbf{Comandos ejecutados:}
\begin{Verbatim}
# En h4 (servidor):
mininet> xterm h4
h4> iperf -s -p 5201

# En h2 (cliente):
mininet> xterm h2
h2> iperf -c 10.0.0.4 -p 5201 -t 10
\end{Verbatim}

\textbf{Resultado obtenido:}

\begin{table}[H]
\centering
\renewcommand{\arraystretch}{1.3}
\begin{tabular}{lccc}
\toprule
\textbf{Lado} & \textbf{Intervalo (s)} & \textbf{Transferencia} & \textbf{Ancho de Banda} \\
\midrule
Servidor (h4) & 0.0000-10.0003 & 11.0 GBytes & 9.43 Gbits/sec \\
Cliente (h2) & 0.0000-10.0107 & 11.0 GBytes & 9.42 Gbits/sec \\
\bottomrule
\end{tabular}
\caption{Resultados Test 5: \texttt{TCP} \texttt{h2} $\rightarrow$ \texttt{h4} (sin firewall en el camino)}
\label{tab:test5_results}
\end{table}

\textbf{Análisis:} 
\begin{itemize}
    \item  Throughput extremadamente alto (~9.4 Gbps) característico de entorno virtualizado
    \item  Transferencia simétrica entre cliente y servidor (11.0 GBytes en ambos lados)
    \item  Variación mínima entre lecturas de cliente y servidor (9.42 vs 9.43 Gbps)
    \item  Este resultado establece la línea base para comparación con tráfico filtrado
\end{itemize}

\subsubsection{Test 6: Throughput TCP Con Firewall Activo (h2 $\rightarrow$ h3)}

\textbf{Objetivo:} Medir el impacto del procesamiento de reglas de firewall en el throughput.

\textbf{Configuración:}
\begin{itemize}
    \item Servidor: \texttt{h3} (puerto \texttt{5201})
    \item Cliente: \texttt{h2}
    \item Duración: 10 segundos
    \item Protocolo: \texttt{TCP}
    \item Nota: El switch \texttt{s2} (firewall) está en el camino, pero el puerto \texttt{5201} NO está bloqueado
\end{itemize}

\textbf{Comandos ejecutados:}
\begin{Verbatim}
# En h3 (servidor):
h3> iperf -s -p 5201

# En h2 (cliente):
h2> iperf -c 10.0.0.3 -p 5201 -t 10
\end{Verbatim}

\textbf{Resultado obtenido:}

\begin{table}[H]
\centering
\renewcommand{\arraystretch}{1.3}
\begin{tabular}{lccc}
\toprule
\textbf{Lado} & \textbf{Intervalo (s)} & \textbf{Transferencia} & \textbf{Ancho de Banda} \\
\midrule
Servidor (h3) & 0.0000-10.0002 & 10.7 GBytes & 9.23 Gbits/sec \\
Cliente (h2) & 0.0000-10.0097 & 10.7 GBytes & 9.22 Gbits/sec \\
\bottomrule
\end{tabular}
\caption{Resultados Test 6: \texttt{TCP} \texttt{h2} $\rightarrow$ \texttt{h3} (con firewall activo)}
\label{tab:test6_results}
\end{table}

\textbf{Comparación con Test 5:}

\begin{table}[H]
\centering
\renewcommand{\arraystretch}{1.3}
\begin{tabular}{lccc}
\toprule
\textbf{Métrica} & \textbf{Sin Firewall (h4)} & \textbf{Con Firewall (h3)} & \textbf{Diferencia} \\
\midrule
Throughput & 9.42 Gbps & 9.22 Gbps & -0.20 Gbps (-2.1\%) \\
Transferencia & 11.0 GBytes & 10.7 GBytes & -0.3 GBytes (-2.7\%) \\
\bottomrule
\end{tabular}
\caption{Comparación de rendimiento: con y sin firewall}
\label{tab:firewall_impact}
\end{table}

\textbf{Análisis:}
\begin{itemize}
    \item  Degradación mínima del rendimiento: ~2\% de reducción en throughput
    \item  La diferencia es estadísticamente pequeña y puede atribuirse a:
    \begin{itemize}
        \item Procesamiento adicional de matching de paquetes contra reglas
        \item Posible diferencia en rutas de red (\texttt{h4} vs \texttt{h3} en topología)
        \item Variabilidad inherente del entorno virtualizado
    \end{itemize}
    \item Conclusión: El firewall SDN tiene impacto despreciable en throughput para tráfico permitido
\end{itemize}

\subsubsection{Test 7: Throughput UDP (h2 $\rightarrow$ h4)}

\textbf{Objetivo:} Evaluar rendimiento de tráfico UDP no bloqueado.

\textbf{Configuración:}
\begin{itemize}
    \item Servidor: \texttt{h4} (puerto \texttt{5201})
    \item Cliente: \texttt{h2}
    \item Duración: 10 segundos
    \item Protocolo: \texttt{UDP}
    \item Bandwidth objetivo: 100 Mbps
\end{itemize}

\textbf{Comandos ejecutados:}
\begin{Verbatim}
# En h4 (servidor):
h4> iperf -s -u -p 5201

# En h2 (cliente):
h2> iperf -c 10.0.0.4 -u -p 5201 -t 10 -b 100M
\end{Verbatim}

\textbf{Resultado obtenido:}

\begin{table}[H]
\centering
\renewcommand{\arraystretch}{1.3}
\begin{tabular}{lccccc}
\toprule
\textbf{Lado} & \textbf{Intervalo} & \textbf{Transfer} & \textbf{Bandwidth} & \textbf{Jitter} & \textbf{Lost/Total} \\
\midrule
Servidor & 0.0-9.9997s & 125 MBytes & 105 Mbps & 0.004 ms & 0/89169 (0\%) \\
Cliente & 0.0-10.0001s & 125 MBytes & 105 Mbps & - & - \\
\bottomrule
\end{tabular}
\caption{Resultados Test 7: \texttt{UDP} \texttt{h2} $\rightarrow$ \texttt{h4}}
\label{tab:test7_results}
\end{table}

\textbf{Análisis:}
\begin{itemize}
    \item  Throughput alcanzado: 105 Mbps (objetivo: 100 Mbps) - ligeramente superior debido al overhead de \texttt{UDP}
    \item  \textbf{0\% de pérdida de paquetes} (0 de 89169 datagramas perdidos)
    \item  Jitter extremadamente bajo: 0.004 ms (excelente para aplicaciones en tiempo real)
    \item  Cliente envió 89169 datagramas, servidor recibió exactamente 89169
    \item  Conclusión: La red virtualizada y el firewall SDN manejan tráfico \texttt{UDP} sin degradación
\end{itemize}

\subsubsection{Test 8: Verificación de Bloqueo con iperf (Puerto 80 TCP)}

\textbf{Objetivo:} Confirmar que el firewall bloquea correctamente el tráfico en puerto restringido usando \texttt{iperf}.

\textbf{Configuración:}
\begin{itemize}
    \item Servidor: \texttt{h3} (puerto 80 - \textbf{BLOQUEADO})
    \item Cliente: \texttt{h2}
    \item Protocolo: \texttt{TCP}
    \item Regla activa: Block port 80 (\texttt{TCP})
\end{itemize}

\textbf{Comandos ejecutados:}
\begin{Verbatim}
# En h3 (servidor):
h3> iperf -s -p 80

# En h2 (cliente):
h2> iperf -c 10.0.0.3 -p 80 -t 5
\end{Verbatim}

\textbf{Resultado obtenido:}

\begin{log}
\begin{Sbox}
\begin{minipage}{0.95\linewidth}
\begin{verbatim}
# Terminal 1 (h3):
Server listening on TCP port 80
TCP window size: 85.3 KBytes (default)

# Terminal 2 (h2):
Client connecting to 10.0.0.3, TCP port 80
TCP window size: 85.3 KBytes (default)

[  1] local 0.0.0.0 port 0 connected with 10.0.0.3 port 80
Connection failed: Connection timed out
\end{verbatim}
\end{minipage}
\end{Sbox}
\fbox{\TheSbox}
\caption{Resultado del intento de conexión \texttt{iperf} al puerto 80 bloqueado}
\label{lab:test8_result}
\end{log}

\textbf{Análisis:}
\begin{itemize}
    \item  El cliente intentó conectarse pero recibió ``Connection timed out"
    \item  El servidor nunca recibió la conexión (no aparece línea ``connected with")
    \item  El firewall descartó los paquetes \texttt{SYN} del \texttt{TCP} handshake\cite{tcp_handshake}
    \item  Esto confirma que la regla de bloqueo del puerto 80 \texttt{TCP} funciona correctamente
    \item  Comportamiento esperado: timeout en lugar de ``Connection refused" porque los paquetes son silenciosamente descartados (\texttt{DROP}) en lugar de rechazados con \texttt{RST}
\end{itemize}

\subsection{Pruebas de Latencia}

\subsubsection{Test 9: Latencia ICMP Entre Hosts Permitidos (h2 $\rightarrow$ h4)}

\textbf{Objetivo:} Medir la latencia introducida por el protocolo ICMP\cite{icmp_rfc} en el camino de datos del controlador SDN.

\textbf{Comando ejecutado:}
\begin{Verbatim}
mininet> h2 ping -c 100 10.0.0.4
\end{Verbatim}

\textbf{Resultado obtenido:}
\begin{log}
\begin{Sbox}
\begin{minipage}{0.95\linewidth}
\begin{verbatim}
PING 10.0.0.4 (10.0.0.4) 56(84) bytes of data.
64 bytes from 10.0.0.4: icmp_seq=1 ttl=54 time=0.875 ms
64 bytes from 10.0.0.4: icmp_seq=2 ttl=54 time=0.115 ms
64 bytes from 10.0.0.4: icmp_seq=3 ttl=54 time=0.056 ms
64 bytes from 10.0.0.4: icmp_seq=4 ttl=54 time=0.065 ms
64 bytes from 10.0.0.4: icmp_seq=5 ttl=54 time=0.060 ms
64 bytes from 10.0.0.4: icmp_seq=6 ttl=54 time=0.070 ms
64 bytes from 10.0.0.4: icmp_seq=7 ttl=54 time=0.064 ms
64 bytes from 10.0.0.4: icmp_seq=8 ttl=54 time=0.072 ms
...
64 bytes from 10.0.0.4: icmp_seq=100 ttl=54 time=0.064 ms

--- 10.0.0.4 ping statistics ---
100 packets transmitted, 100 received, 0% packet loss, time 110574ms
rtt min/avg/max/mdev = 0.046/0.072/0.876/0.081 ms
\end{verbatim}
\end{minipage}
\end{Sbox}
\fbox{\TheSbox}
\caption{Resultado del ping ICMP entre \texttt{h2} y \texttt{h4}}
\label{lab:test9_result}
\end{log}


\textbf{Análisis de Resultados:}

\begin{table}[H]
\centering
\renewcommand{\arraystretch}{1.3}
\begin{tabular}{lcc}
\toprule
\textbf{Métrica} & \textbf{Valor} & \textbf{Observación} \\
\midrule
RTT mínimo & 0.046 ms & Excelente para red virtualizada \\
RTT promedio & 0.072 ms & Muy bajo, típico de entorno local \\
RTT máximo & 0.876 ms & Pico en primer paquete (icmp\_seq=1) \\
Desviación estándar & 0.081 ms & Baja variabilidad \\
Pérdida de paquetes & 0\% & Sin pérdidas (100/100) \\
Tiempo total & 110.574 s & ~1.1s por paquete (esperado) \\
\bottomrule
\end{tabular}
\caption{Estadísticas de latencia \texttt{ICMP} (\texttt{h2} $\rightarrow$ \texttt{h4})}
\label{tab:latency_stats}
\end{table}

\textbf{Observaciones importantes:}

\begin{enumerate}
    \item \textbf{Primer paquete con alta latencia (0.875 ms):}
    \begin{itemize}
        \item Este es el comportamiento esperado en SDN
        \item El primer paquete genera un PacketIn al controlador
        \item El controlador procesa, instala reglas de forwarding, y envía PacketOut
        \item Latencia adicional: ~0.8 ms (0.875 - 0.070 promedio)
        \item Este overhead solo ocurre una vez por flujo nuevo
    \end{itemize}
    
    \item \textbf{Paquetes subsiguientes (0.046 - 0.072 ms):}
    \begin{itemize}
        \item Procesados directamente por el switch sin consultar al controlador
        \item Latencia normal de conmutación en entorno virtualizado
        \item Consistencia alta (desviación estándar 0.081 ms)
    \end{itemize}
    
    \item \textbf{TTL = 54:}
    \begin{itemize}
        \item Indica que los paquetes atravesaron múltiples switches (\texttt{TTL} inicial típicamente 64)
        \item Decremento de ~10 hops sugiere topología en cascada
        \item Compatible con 4 switches + procesamiento del kernel
    \end{itemize}
    
    \item \textbf{0\% packet loss:}
    \begin{itemize}
        \item Demuestra estabilidad del sistema
        \item No hay saturación de tablas de flujos
        \item Controlador responde a tiempo a todos los PacketIn
    \end{itemize}
\end{enumerate}

\textbf{Comparación con redes reales:}

\begin{table}[H]
\centering
\renewcommand{\arraystretch}{1.3}
\begin{tabular}{lcc}
\toprule
\textbf{Tipo de Red} & \textbf{RTT Típico} & \textbf{Comparación} \\
\midrule
Mininet (este test) & 0.072 ms & - \\
LAN Ethernet física & 0.2 - 1 ms & 3-14x más lento \\
WAN inter-ciudad & 10 - 50 ms & 140-700x más lento \\
Internet global & 100 - 300 ms & 1400-4200x más lento \\
\bottomrule
\end{tabular}
\caption{Comparación de latencias entre diferentes tipos de red}
\label{tab:latency_comparison}
\end{table}

\textbf{Conclusión Test 9:}
\begin{itemize}
    \item  La latencia promedio de 0.072 ms es excelente para una red SDN virtualizada
    \item  El overhead del controlador (0.8 ms en primer paquete) es aceptable
    \item  La estabilidad y consistencia demuestran que el firewall no introduce jitter significativo
    \item  El sistema es adecuado para aplicaciones sensibles a latencia en entornos de data center
\end{itemize}

\subsection{Análisis de Flows Instalados}

\subsubsection{Test 10: Verificación de Reglas Proactivas}

\textbf{Objetivo:} Confirmar que las reglas de firewall se instalan proactivamente al conectar el switch.

\textbf{Comando ejecutado:}
\begin{Verbatim}
mininet> sh ovs-ofctl dump-flows s2
\end{Verbatim}

\textbf{Resultado obtenido:}
\begin{log}
\begin{Sbox}
\begin{minipage}{0.95\linewidth}
\begin{verbatim}
cookie=0x0, duration=45.123s, table=0, n_packets=0, n_bytes=0, 
    priority=10000,tcp,tp_dst=80 actions=drop
cookie=0x0, duration=45.122s, table=0, n_packets=0, n_bytes=0, 
    priority=10000,udp,tp_dst=80 actions=drop
cookie=0x0, duration=45.121s, table=0, n_packets=0, n_bytes=0, 
    priority=10000,sctp,tp_dst=80 actions=drop
cookie=0x0, duration=45.120s, table=0, n_packets=0, n_bytes=0, 
    priority=10000,udp,nw_src=10.0.0.1,tp_dst=5001 actions=drop
cookie=0x0, duration=45.119s, table=0, n_packets=0, n_bytes=0, 
    priority=10000,dl_src=00:00:00:00:00:01,dl_dst=00:00:00:00:00:03 
    actions=drop
cookie=0x0, duration=45.118s, table=0, n_packets=0, n_bytes=0, 
    priority=10000,dl_src=00:00:00:00:00:03,dl_dst=00:00:00:00:00:01 
    actions=drop
\end{verbatim}
\end{minipage}
\end{Sbox}
\fbox{\TheSbox}
\caption{Flujos instalados en el switch \texttt{s2} (firewall)}
\label{lab:test10_result}
\end{log}

\textbf{Análisis:}
\begin{itemize}
    \item  Se observan 6 reglas instaladas con prioridad 10000 (firewall)
    \item  Las reglas incluyen matching por protocolo (\texttt{tcp/udp/sctp}), puerto (80, 5001), y direcciones \texttt{MAC}
    \item  Todas las reglas tienen acción ``\texttt{drop}''
    \item  El campo \texttt{n\_packets=0} indica que las reglas están instaladas pero aún no han procesado tráfico bloqueado
    \item  Los timestamps (duration) muestran que todas se instalaron en secuencia rápida (~0.001s entre cada una)
\end{itemize}

\subsection{Conclusiones de las Pruebas}

Las pruebas realizadas demuestran que el sistema de firewall SDN implementado cumple con todos los requisitos funcionales y de rendimiento:

\begin{enumerate}
    \item \textbf{Corrección funcional:} Todas las reglas de firewall se aplican correctamente, bloqueando tráfico específico por protocolo, puerto, \texttt{IP} y \texttt{MAC}.
    
    \item \textbf{Rendimiento TCP:} Throughput de ~9.4 Gbps sin firewall vs ~9.2 Gbps con firewall (\textbf{degradación < 3\%}), demostrando impacto mínimo en tráfico permitido.
    
    \item \textbf{Rendimiento UDP:} 105 Mbps sostenido con \textbf{0\% pérdida} de paquetes y jitter de 0.004 ms, ideal para aplicaciones en tiempo real.
    
    \item \textbf{Latencia:} RTT promedio de 0.072 ms con primer paquete en 0.875 ms (overhead del controlador SDN aceptable).
    
    \item \textbf{Granularidad:} El sistema permite reglas específicas (ej: solo h1 bloqueado para UDP:5001) y genéricas (ej: bloquear puerto 80 para todos).
    
    \item \textbf{Bidireccionalidad:} El bloqueo bidireccional entre hosts funciona independientemente de la dirección del tráfico.
    
    \item \textbf{Instalación proactiva:} Las 6 reglas de firewall se instalan automáticamente al conectar el switch.
    
    \item \textbf{Estabilidad:} 0\% pérdida de paquetes en 100 pings, demostrando que el sistema no introduce inestabilidad.
\end{enumerate}

\textbf{Limitaciones observadas:}
\begin{itemize}
    \item Latencia adicional del primer paquete (~0.8 ms) debido al procesamiento del controlador
    \item Throughput ligeramente menor cuando el tráfico atraviesa el switch de firewall (\texttt{s2})
    \item Dependencia de un controlador centralizado (single point of failure)
\end{itemize}

\textbf{Casos de uso validados:}
\begin{itemize}
    \item  Bloqueo de servicios (\texttt{HTTP} puerto 80)
    \item  Aislamiento de hosts problemáticos (\texttt{h1} $\leftrightarrow$ \texttt{h3})
    \item  Filtrado selectivo por \texttt{IP} y puerto (\texttt{h1} $\rightarrow$ \texttt{UDP:5001})
    \item  Políticas genéricas multi-protocolo (puerto 80 en \texttt{TCP/UDP/SCTP})
\end{itemize}