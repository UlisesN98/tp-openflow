\section{Introducción}
\label{sec:introduccion}

En la actualidad, las redes de computadoras han evolucionado significativamente hacia arquitecturas más flexibles y programables. Históricamente, los switches se han comportado como dispositivos estáticos, con lógica de encaminamiento fija implementada en hardware o firmware propietario. Sin embargo, el surgimiento de \textbf{Software-Defined Networking (SDN)} y el protocolo \textbf{OpenFlow} ha revolucionado la forma en que se controla y administra el comportamiento de la red.

OpenFlow es un protocolo de comunicación abierto que permite que un controlador centralizado determine dinámicamente cómo los switches deben procesar y reenviar paquetes de datos. Esta separación entre el plano de datos (switches) y el plano de control (controlador) proporciona una mayor flexibilidad, simplificando la administración de redes y permitiendo la implementación de políticas de red complejas.

El presente trabajo práctico implementa un \textbf{Firewall} utilizando OpenFlow 1.0 simplificado utilizando el framework \textbf{POX}, un controlador escrito en Python que proporciona una plataforma accesible para comprender los principios de SDN. Se desarrolló una topología de red personalizada en \textbf{Mininet} con múltiples switches conectados en cascada y hosts distribuidos, permitiendo experimentar con el comportamiento del plano de datos cuando es controlado por un controlador centralizado.

Este informe detalla el diseño e implementación de la topología de red, la arquitectura del controlador POX, las funcionalidades desarrolladas, y los resultados obtenidos a través de diferentes escenarios de prueba. Se analizan aspectos como la comunicación entre controlador y switches, el aprendizaje de direcciones MAC, y el comportamiento de la red bajo diferentes cargas y condiciones.

\subsection{Objetivos}
Los objetivos principales de este trabajo son:
\begin{itemize}
    \item Comprender los fundamentos de \textbf{Software-Defined Networking (SDN)} y el protocolo \textbf{OpenFlow 1.0}.
    
    \item Implementar un \textbf{controlador SDN} utilizando el framework \textbf{POX} en Python, capaz de gestionar el comportamiento de switches virtuales.
    
    \item Diseñar y desplegar una \textbf{topología de red personalizada} en \textbf{Mininet} con múltiples switches conectados en cascada y hosts distribuidos.
    
    \item Desarrollar un \textbf{firewall proactivo} que instale reglas de filtrado al momento de conexión de los switches, bloqueando tráfico según:
    \begin{itemize}
        \item Direcciones MAC (capa 2)
        \item Direcciones IP y protocolo ICMP (capa 3)
        \item Protocolos de transporte: TCP, UDP y SCTP (capa 4)
        \item Puertos de origen y destino
    \end{itemize}
    
    \item Implementar un mecanismo de \textbf{aprendizaje de direcciones MAC} (MAC learning) para el reenvío inteligente de paquetes entre hosts.
    
    \item Diseñar un sistema de \textbf{reglas configurables} mediante archivos \texttt{JSON} que permitan especificar políticas de firewall sin modificar el código del controlador.
    
    \item Soportar la expansión automática de reglas genéricas (protocolo "ANY") en múltiples reglas específicas para cada protocolo de transporte.
    
    \item Implementar detección reactiva de paquetes bloqueados, instalando dinámicamente reglas DROP cuando un paquete llega al controlador y matchea con las políticas del firewall.
\end{itemize}